\section{Системы линейных алгебраических уравнений (СЛАУ)}

\begin{definition}
  \textit{Системой линейных алгебраических уравнений} называется система уравнений вида:
  \begin{align*}
    &a_{11} x_1  + a_{12} x_2 + \ldots a_{1n} = b_1 \\
    &a_{21} x_1  + a_{22} x_2 + \ldots a_{2n} = b_2 \\
    &\ldots \\
    &a_{m1} x_1  + a_{m2} x_2 + \ldots a_{mn} = b_m \\
  \end{align*}
  где $a_{ij} = const$, $i=1..m$, $j=1..n$ -- коэффициенты СЛАУ, $b_i = const$ -- свободный член СЛАУ,  $x_{i}, i=1..n$ -- неизвестная переменная СЛАУ.
\end{definition}

\begin{definition}
  Совокупность переменных $(x_1, x_2 \ldots x_{n})$, при которых каждое уравнение обращается в верное равенство, называется \textit{решением} данной СЛАУ. 
\end{definition}

Форма записи СЛАУ выше называется \textit{координатной}. 

\subsubsection*{Матричная форма}

Обозначим:
\begin{gather*}
  A = \left( 
\begin{matrix}
  a_{11} & a_{12} & \ldots & a_{1n} \\
  a_{21} & a_{22} & \ldots & a_{2n} \\
  \ldots & \ldots & \ldots & \ldots \\
  a_{m1} & a_{m2} & \ldots & a_{mn} \\
\end{matrix}
  \right) \\
  B = \left( 
    \begin{matrix}
      b_1 \\
      b_2 \\
      \ldots \\
      b_m
    \end{matrix}
  \right) \qquad
  X = \left( 
  \begin{matrix}
    x_1 \\
    x_2 \\
    \ldots \\
    x_n
  \end{matrix}
  \right) 
\end{gather*}

Тогда СЛАУ можно записать в виде: \[
  \boxed{A \cdot X = B}
\] 

\subsubsection*{Векторная форма записи}

Обозначим:
\begin{gather*}
  \vec{a}_1 = \left( 
    \begin{matrix}
      a_{11} \\
      a_{21} \\
      \ldots \\
      a_{m1} \\
    \end{matrix}
  \right) \qquad
  \vec{a}_2 = \left( 
    \begin{matrix}
      a_{21} \\
      a_{22} \\
      \ldots \\
      a_{2m}
    \end{matrix}
  \right) \qquad \ldots \qquad 
  \vec{a}_n = \left( 
    \begin{matrix}
      a_{n1} \\
      a_{n2} \\
      \ldots \\
      a_{nm}
    \end{matrix}
  \right) \qquad 
  \vec{b} = \left( 
    \begin{matrix}
      b_1 \\ 
      b_2 \\
      \ldots \\
      b_m
    \end{matrix}
  \right)
\end{gather*}

Тогда вектор $\vec{b}$, координаты которого являются свободные члены, можно представить в виде линейной комбинаций векторов $\vec{a}$, координаты которых соответствуют элементам столбцов матрицы.

\begin{definition}
  СЛАУ, имеющая решение, назыается \textit{совместной}.
\end{definition}

\begin{definition}
  СЛАУ, не имеющая решение, называется \textit{несовместной}.  
\end{definition}

\begin{definition}
  Совместная СЛАУ, имеющая единственное решение, называется \textit{совместно-определённой}. 
\end{definition}

\begin{definition}
  Совместная СЛАУ, имеющая бесконечное кол-во решений, называется \textit{совместно-неопределённой}.
\end{definition}

\begin{definition}
  СЛАУ, у которой все свободные члены равны нулю, называется \textit{однородной}.
\end{definition}

\begin{definition}
  СЛАУ, у которой хотя бы один свободный член не равен нулю, называется \textit{неоднородной}.
\end{definition}

\subsection{Решение матричных уравнений}

\subsection*{I}

Для уравнения вида: \[
A \cdot X = B
\] 
\begin{enumerate}
  \item Умножим обе части уравнения на обратную матрицу $A$ \textbf{слева}:
    \begin{align*}
      A^{-1} \cdot A \cdot X &= A^{-1} \cdot B \\
      E \cdot X &= A^{-1} \cdot B \\
      X &= A^{-1} \cdot  B
    \end{align*}
\end{enumerate}

\subsection*{II}

Для уравнений вида: \[
X \cdot A = B
\] 
\begin{enumerate}
  \item Умножим обе части уравнения на обратную матрицу $A$ \textbf{справа}:
    \begin{align*}
      X \cdot A \cdot A^{-1} &= B \cdot A^{-1} \\
      X \cdot E &= B \cdot A^{-1} \\
      X &= B \cdot A^{-1} \\
    \end{align*}
\end{enumerate}

\subsection*{III}

Для уранвения вида: \[
A \cdot X \cdot C = B
\] 
\begin{enumerate}
  \item Умножим обе части уравнения на обратную матрицу $A$ \textbf{слева} и на обратную матрицу $C$ \textbf{справа}:
  \begin{align*}
    A^{-1} \cdot  A \cdot  X \cdot C \cdot C^{-1} &= A^{-1} \cdot B \cdot C^{-1} \\
    E \cdot  X \cdot E &= A^{-1} \cdot B \cdot C^{-1} \\
    X &= A^{-1} \cdot B \cdot C^{-1}
  \end{align*}
\end{enumerate}

\subsection{Формулы Крамера для решения СЛАУ}

Запишем СЛАУ в матричном виде: 
\begin{gather*}
  A \cdot X = B \qquad A_{n \times n} \\
  A = \left( 
  \begin{matrix}
    a_{11} & a_{12} & \ldots & a_{1n} \\
    a_{21} & a_{22} & \ldots & a_{2n} \\
    \ldots & \ldots & \ldots & \ldots \\
    a_{n1} & a_{n2} & \ldots & a_{nn}
  \end{matrix}
  \right) 
\end{gather*}

Пусть матрица не вырожденная. Тогда её обратная матрица будет иметь вид: 
\begin{gather*}
  A^{-1} = \frac{1}{det A} \left( 
    \begin{matrix}
    A_{11} & A_{12} & \ldots & A_{1n} \\
    A_{21} & A_{22} & \ldots & A_{2n} \\
    \ldots & \ldots & \ldots & \ldots \\
    A_{n1} & A_{n2} & \ldots & A_{nn}
    \end{matrix}
  \right)^{\tau} = \left( 
  \begin{matrix}
    \frac{A_{11}}{det A} & \frac{A_{12}}{det A} & \ldots & \frac{A_{1n}}{det A} \\
    \frac{A_{21}}{det A} & \frac{A_{22}}{det A} & \ldots & \frac{A_{2n}}{det A} \\
    \ldots & \ldots & \ldots & \ldots \\
    \frac{A_{n1}}{det A} & \frac{A_{n2}}{det A} & \ldots & \frac{A_{nn}}{det A}
  \end{matrix}
  \right) \\
  \\
  X = \left( 
  \begin{matrix}
    \frac{A_{11}}{det A} & \frac{A_{12}}{det A} & \ldots & \frac{A_{1n}}{det A} \\
    \frac{A_{21}}{det A} & \frac{A_{22}}{det A} & \ldots & \frac{A_{2n}}{det A} \\
    \ldots & \ldots & \ldots & \ldots \\
    \frac{A_{n1}}{det A} & \frac{A_{n2}}{det A} & \ldots & \frac{A_{nn}}{det A}
  \end{matrix}\right) \cdot \left( 
  \begin{matrix}
    b_1 \\
    b_2 \\
    \ldots \\
    b_{n}
  \end{matrix}
\right) \\
  \\
  x_i = 
  \frac{A_{i1}}{det A} b_1 +
  \frac{A_{i2}}{det A} b_2 +
  \ldots +
  \frac{A_{in}}{det A} b_n = \\
  = \frac{A_{i1}b_1 + A_{i_2}b_2 + \ldots + A_{in}b_n}{det A}
\end{gather*}

Заметим, что числитель последнего выражения это ничто иное, как:
\begin{gather*}
  \Delta_1 = 
  \begin{vmatrix} 
    b_1 & a_{12} & a_{13} & \ldots a_{1n} \\
    b_2 & a_{22} & a_{23} & \ldots a_{2n} \\
    \ldots \\
    b_n & a_{2n} & a_{n3} & \ldots a_{nn} \\
  \end{vmatrix} \\
  \text{ для } i = 1
\end{gather*}

\begin{note}
  Определитель $\Delta_i$ получается, если элементы i-ного столбца заменить на свободные члены СЛАУ.
\end{note}

Если квадратная матрица невырожденная, то однородная СЛАУ имеет \textit{единственное решение}. 

Если квадратная матрица вырожденная, то однородная СЛАУ имеет \textit{бесконечное количество решений}. 

