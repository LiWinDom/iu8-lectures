\section{Кривые второго порядка}

Общее уравнение кривой второго порядка:
\[
  \boxed{x^2 + 2Bxy + Cy^2 + Dx + Ey + F = 0}
\] 
где:
\begin{gather*}
  A, B, C, D, E, F = const \\
  A^2 + B^2 + C^2 > 0
\end{gather*}

\subsection{Эллипс}

\begin{definition}
  \textit{Эллипсом} называется геометрическое место точек, сумма расстояний от каждой из которых до двух фиксированных точек, называемых \textit{фокусами}, постоянна и равна $2a$.
\end{definition}

$F_1, F_2$ - фокусы эллипса

Расстояние между фокусами называется \textit{фокальным расстоянием}.

Расстояние от каждой точки эллипса до фокуса называется \textit{фокальным радиусом} 

Прямая, которая проходит через фокусы, и прямая, которая проходит через середину этой прямой и перпендикулярной ей, являются \textit{осями симметрии данного эллипса}.
Первая прямая называется \textit{большой осью}, а вторая -- \textit{малой осью}.   

Точка пересечения осей эллипса называется \textit{центром эллипса}, а точки пересечения эллипса с осями называются \textit{вершинами эллипса}. 

\subsubsection*{Уравнение эллипса}

Расположим прямоугольную систему координат так, чтобы её начало совпадало с центром эллипса, а фокусы лежали на оси абцисс.

$O$ -- центр эллипса \\
 $F_1, F_2$ -- фокусы эллипса \\
 $A_1, A_2, A_3, A_4$ -- вершины эллипса \\
$F_1F_2 = 2c$  -- фокусное (фокальное) расстояние\\
Возьмём точку $M(x, y)$, принадлежащей эллипсу, и составим векторы:
\begin{gather*}
  \overrightarrow{F_1M} = \{x + c, y\} \\
  \overrightarrow{F_2M} = \{x - c, y\}
\end{gather*}
Тогда:
\begin{gather*}
  |\overrightarrow{F_1M}| + |\overrightarrow{F_2M}| = 2a \\
  \sqrt{(x + c)^2 + y^2} + \sqrt{(x - c)^2 + y^2} = 2a \\
  \sqrt{(x + c)^2 + y^2} = 2a - \sqrt{(x - c)^2 + y^2} \\
  (x + c)^2 + y^2 = 4a^2 - 4a\sqrt{(x - c)^2 + y^2} + x^2 - 2xc + c^2 + y^2 \\
  4a\sqrt{(x - c)^2 + y^2} = 4a^2 - 4xc \\
  a\sqrt{(x - c)^2 + y^2} = a^2 - xc \\
  x^2 a^2 - 2a^2xc + a^2y^2 = a^4 - 2a^2xc + x^2c^2 \\
  x^2a^2 + x^2c^2 + a^2y^2 = a^4 - a^2c^2 \\
  x^2(a^2 - c^2) + a^2y^2 = a^2(a^2 - c^2) \\
  \frac{x^2}{a^2} + \frac{y^2}{a^2 - c^2} = 1
\end{gather*} 

Обозначим $b^2 = a^2 - c^2$. Получаем \textit{каноническое уравнение эллипса}: \[
  \boxed{\frac{x^2}{a^2} + \frac{y^2}{b^2} = 1}
\]  
где $a$ -- \textit{большая полуось эллипса}, а  $b$ -- \textit{малая полуось эллипса}.


\\
Отношение фокусного расстояния эллипса к большой оси называется \textit{эксцентриситетом} эллипса.
\begin{gather*}
  \frac{F_1F_2}{A_3A_1} = \frac{2c}{2a} = \varepsilon \\
  \boxed{\varepsilon = \frac{c}{a}}
\end{gather*}
\begin{note}
  Т.к. $a < c$, то $0 < \varepsilon < 1$
\end{note}

Центриситет показывает степень "сжатия" эллипса.

Отношение фокального радиуса точки эллипса к расстоянию до некоторой прямой, называемой \textit{директрисой}, постоянно и равно \textit{эксцентриситету}. 

Уравнение директрис:
\begin{align*}
  &d_1: x = -\frac{a}{\varepsilon} \\
  &d_2: x = \frac{a}{\varepsilon}
\end{align*}

\begin{note}
  \begin{center}
    \begin{enumerate}
      \item Уравнение эллипса с центром в точке $O(x_0, y_0)$: \[
        \frac{(x - x_0)^2}{a^2} + \frac{(y - y_0)^2}{b^2} = 1
        \]

      \item Уравнение мнимого эллипса с центром в точке $O(0, 0)$ \[
        \frac{x^2}{a^2} + \frac{y^2}{b^2} = -1
      \]  

      \item Если $a = b = R$, то это уравнение окружности:
        \begin{gather*}
          \frac{x^2}{R^2} + \frac{y^2}{R^2} = 1 \\
          x^2 + y^2 = R^2
        \end{gather*}
        Для окружности в точке $O(x_0, y_0)$: \[
          (x - x_0)^2 + (y - y_0)^2 = R^2
        \]

      \item Если $a < b$, то изображение эллипса "переворачивается" на 90:
  \end{enumerate} 
  \end{center}
\end{note}

\subsection{Гипербола}

\begin{definition}
  \textit{Гиперболой} называется геометрическое место точек, разность расстояний от каждой их которых до двух фиксированных точек, называемых \textit{фокусами}, постоянно и равно $2a$.
\end{definition}

Прямая, на которой лежат фокусы, и прямая, которая проходит через середину отрезка, соединяющего фокусы и перпендикулярная ей, называеются \textit{осями симметрии гиперболы}. 
Первая прямая называется \textit{действительной осью}, а вторая -- \textit{мнимой осью}.

$F_1, F_2$ -- фокусы \\
$F_1F_2 = 2c$ -- фокусное (фокальное) расстояние

Точки пересечения действительной и мнимой оси гиперболы называется \textit{центром гиперболы}, а точка пересечения с \underline{действительной осью} называются \textit{вершинами гиперболы}.

\subsection*{Уравнение гиперболы}

Расположим декартову систему координат так, чтобы её начало совпадало с центром гипероболы, а фокусы лежали на оси абцисс.

$F_1(-c, 0), F_2(c, 0)$.

Возьмём произвольную точку $M(x, y)$, принадлежащей гиперболе.
\begin{gather*}
  \overrightarrow{F_1M} = \{x + c, y\} \\
  \overrightarrow{F_2M} = \{x - c, y\} \\
  |\overrightarrow{F_1M}| - |\overrightarrow{F_2M}| = 2a \\
  \sqrt{(x + c)^2 + y^2} - \sqrt{(x - c)^2 + y^2} = 2a \\
  \sqrt{(x + c)^2 + y^2} = 2a + \sqrt{(x - c)^2 + y^2} \\
  (x + c)^2 + y^2 = 4a^2 + 4a\sqrt{(x - c)^2 + y^2} + (x - c)^2 + y^2 \\
  4a\sqrt{(x - c)^2 + y^2} = 4xc - 4a^2 \\
  a\sqrt{(x - c)^2 + y^2} = xc - a^2 \\
  x^2 a^2 - 2a^2xc + a^2y^2 = x^2c^2 - 2a^2xc + a^4 \\
  x^2a^2 + x^2c^2 + a^2y^2 = a^4 - a^2c^2 \\
  x^2(a^2 + c^2) + a^2y^2 = a^2(a^2 - c^2) \\
  \frac{x^2}{a^2} + \frac{y^2}{a^2 - c^2} = 1
\end{gather*}

Обозначим $b^2 = a^2 - c^2$. Получаем \textit{каноническое уравнение эллипса}: \[
  \boxed{\frac{x^2}{a^2} - \frac{y^2}{b^2} = 1}
\]  

\textit{Центриситетом эллипса} называется: \[
\varepsilon = \frac{2c}{2a} = \frac{c}{a}
\] 
\begin{note}
  Т.к. $c > a$, то  $\varepsilon > 1$
\end{note}

\begin{note}
  Уравнение сопряжённой гиперболы:
  \begin{gather*}
    \boxed{-\frac{x}{a^2} + \frac{y^2}{b^2} = 1} \text{ или } \boxed{\frac{x^2}{b^2} = -1} 
  \end{gather*} 

  Уравнение гиперболы с центром в точке $M(x_0, y_0)$: \[
  \frac{(x - x_0)}{a^2} - \frac{(y - y_0)}{b^2} = 1
  \] 

  Если $a = b$, то гипербола становится \textit{равносторонней}.

  Если: 
  \begin{gather*}
    \frac{x^2}{a^2} - \frac{y^2}{b^2} = 0
  \end{gather*}
  то получается вырожденной уравнение -- две пересекающиеся прямые:
  \begin{gather*}
    b^2x^2 - a^2y^2 = 0
    (bx - ay)(bx + ay) = 0 \\
    \begin{cases}
      bx - ay = 0 \\
      bx + ay = 0
    \end{cases} \implies 
    \begin{cases}
      y = \frac{b}{a} x \\
      y = - \frac{b}{a} x
    \end{cases}
  \end{gather*}
  Эти же уравненения и являются \textit{уравнениями ассимптот}. 
  
  Если центр гиперболы $O(x_0, y_0)$, то: 
  \begin{gather*}
  y - y_0 = \frac{b}{a}(x - x_0) \implies y = \frac{b}{a}x + (y_0 - \frac{b}{a}x_0 \\
  y - y_0 = -\frac{b}{a}(x - x_0) \implies y = -\frac{b}{a}x + (y_0 + \frac{b}{a}x_0
  \end{gather*}
\end{note}

\subsection{Парабола}

\begin{definition}
  \textit{Параболой} называется геометрическое место точек, расстояние от каждой из которых до некоторой точки, называмой \textit{фокусом}, и фиксированной прямой, называемой \textit{директрисой}, равно.
\end{definition}

\subsection*{Уравнение параболы}

Расположим декартову систему координат так, чтобы начало координат совпадало с вершиной параболы.

\[
A(-\frac{p}{2}), \quad F(\frac{p}{2}, 0)
\] 

\begin{gather*}
  \overrightarrow{AM} = \{x + \frac{p}{2}, a\}, \quad \overrightarrow{FM} = \{x - \frac{p}{2}, y\} \\
  |\overrightarrow{AM}| = |\overrightarrow{FM}| \\
  \sqrt{\left( x + \frac{p}{2} \right)^2 } = \sqrt{\left( (x - \frac{p}{2}) + y^2 \right)} \\
  x^2 + xp + \frac{p^2}{4} = x^2 - xp + \frac{p^2}{4} + y^2 \\
\end{gather*}
Тогда получаем каноническое уравнение параболы с вершиной в $O(0, 0)$:  \[
  \boxed{y^2 = 2px}
\] 

Если $p > 0$, то ветви параболы направлены \textit{вправо}, если  $p < 0$, то ветви направлены  \textit{влево}.

Если вершина в точке $Mx_0, y_0)$, тогда: 
\begin{gather*}
  (y - y_0)^2 = 2p(x - x_0)^2
\end{gather*}

Уравнение директрисы:
\begin{gather*}
  d: x = -\frac{p}{2}
\end{gather*}

\subsection{Примеры}

\begin{eg}
  \begin{gather*}
    2x^2 - 4y^2 - 6x + 8y - 10 = 0 \\
    2(x^2 - 3x) - 4(y^2 - 2y) - 10 = 0 \\
    2(x^2 - 3x + \frac{9}{4} - \frac{9}{4}) - -4(y^2 - 2y + 1 - 1) - 10 = 0 \\
    2(x - \frac{3}{2})^2 - \frac{9}{2} - 4(y - 1)^2 + 4 - 10 \\
    2\left( x - \frac{3}{2} \right) ^2 - 4(y - 1)^2 = \frac{21}{2} \\
    \frac{\left( x - \frac{3}{2} \right)^2 }{\frac{21}{4}} - \frac{\left( y - 1 \right)^2}{\frac{21}{8}} = 1
  \end{gather*}
  Получили \textit{уравнение гиперболы} с центром в $O\left(\frac{3}{2}, 1\right)$, действительная полуось $a = \frac{\sqrt{21}}{2}$ и мнимая полуось $b = \sqrt{\frac{21}{8}}$.
\end{eg}

