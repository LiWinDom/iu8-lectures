\section{Линейная зависимость и независимость векторов}

\begin{definition}
  \begin{gather*}
    \lambda_1 \vec{a_1} + \lambda_2 \vec{a_2} + \ldots + \lambda_n \vec{a_n} \\
    \text{где } \lambda_i - \text{произвольные числа}
  \end{gather*}
  называется линейной комбинацией системы векторов $\vec{a}$, а числа $\lambda$ - коэффициентом линейной комбинации.  \\
\end{definition}

Если $\forall \lambda = 0$, то линейную комбинацию называют \textit{тривиальной}. \\
Если $\neg \forall \lambda = 0$, то линейную комбинацию называют \textit{нетривиальной}. 

\begin{definition}
Система векторов называется \textit{линейно-зависимой}, если существует нетривиальная равная нулевому вектору линейной комбинация этих векторов:
  \begin{gather*}
    \lambda_1 \vec{a_1} + \lambda_2 \vec{a_2} + \ldots + \lambda \vec{a_n} = \vec{0} \\
    \lambda_1^2 + \lambda_2^2 + \ldots + \lambda_n^2 = 0
  \end{gather*}
\end{definition}

\begin{definition}
  Система векторо называется \textit{линейно-независимой}, если существует только тривиальная равная нулевому вектору линейная комбинация.
  \[
    \lambda_1 \vec{a_1} + \lambda_2 \vec{a_2} + \ldots + \lambda \vec{a_n} = \vec{0}
  \] 
\end{definition}

\begin{theorem}
  Система векторов линейно-зависима тогда и только тогда, когда один из этих векторов можно представить в виде линейной комбинации других векторов.
\end{theorem}
\begin{proof}
  1). Пусть система векторов линейно-зависима. \\
  Тогда по определению существует нетривиальная равная нулевому вектору линейная комбинация этих векторов:
  \begin{gather*}
    \lambda_1 \neq  0 \\
    \lambda_1 \vec{a_1} + \lambda_2 \vec{a_2} + \ldots + \lambda_n \vec{a_n} = \vec{0} \\
    \vec{a_1} = - \frac{\lambda_2}{\lambda_1} \vec{a_2} - \frac{\lambda_3}{\lambda_1} \vec{a_3} - \ldots - \frac{\lambda_n}{\lambda_1} \vec{a_n}
  \end{gather*}
  Обозначим $\beta_i = - \frac{\lambda_i}{\lambda_1}$, где $i \in N \land 2 \le i \le n$. \\
  Получаем:
  \[
  \vec{a_1} = \beta_2 \vec{a_2} + \beta_3 \vec{a_3} + \ldots + \beta_n \vec{a_n}
  \] 
  Что и требовалось доказать.
\end{proof}
\begin{proof}
  2) Пусть один из векторов можно представить в виде линейной комбинации другиз векторов системы (возьмем $\vec{a_1}$. Перенесём слагаемые из правой части в левую:
  \[
  \vec{a_1} - \lambda_2 \vec{a_2} - \lambda_3 \vec{a_3} - \ldots - \lambda_n \vec{a_n} = \vec{0}
  \]
  Получили нетривиальную равную нулевому вектору линейную комбинацию векторов. По определению, данная система векторов является \textit{линейно-зависимой}. 
\end{proof}

\subsection{Критерии линейной зависимости 2 и 3 векторов}

\begin{theorem}
  Два вектору \textit{линейно-зависимы} тогда и только тогда, когда они \textit{коллинеарны}.
\end{theorem}
\begin{proof}
  1) Необходимость. \\
Пусть система векторв $\vec{a_1}, \vec{a_2}$ линейно-зависима. Тогда по определению $\exists$ нетривиальная линейная зависимость $=\vec{0}$ этих векторов. Пусть $\lambda_1 \neq 0$, тогда $\vec{a_1} = - \frac{\lambda_2}{\lambda_1} \vec{a_2}$. Обозначим $\beta = -\frac{\lambda_2}{\lambda_1}$, тогда $\vec{a_1} = \beta \vec{a_2}$. По определению произведение вектора на число $\vec{a_1}$ и $\vec{a_2}$ коллинеарны.
  2) Достаточность. \\
  Пусть $\vec{a_1} \parallel \vec{a_2}$. Тогда $ \vec{a_1} = \lambda \vec{a_2}$ (по определению произведения вектора на число). Перенесем все налево: 
  \[
  \vec{a_1} - \lambda \vec{a_2} = \vec{0}
  \] 
  По определению $\vec{a_1}$ и $\vec{a_2}$ являются линейной зависимостью.
\end{proof}

\begin{theorem}
  Три вектора линейной зависимы тогда и только тогда, когда они компланарны.
\end{theorem}
\begin{proof}
  (1) Пусть $\vec{a_1}$, $\vec{a_2}$, $\vec{a_3}$ - линейная зависимость, тогда по определению существуют:
  \[
  \lambda_1 \vec{a_1} + \lambda_2 \vec{a_2} + \lambda_3 \vec{a_3} = \vec{0}
  \] 
  Тогда:
  \begin{gather*}
    \lambda_1 \neq 0 \\
    \vec{a_1} = -\frac{\lambda_2}{\lambda_1} \vec{a_2} - \frac{\lambda_3}{\lambda_1} \vec{a_3}
  \end{gather*}

  Обозначим $\beta = -\frac{\lambda_i}{\lambda}$, где $i = 2, 3$. \[
    \vec{a_1} = \beta_2 \vec{a_2} + \beta_3 \vec{a_3}
  \]
  Совместим начала $\vec{a_2}$ и $\vec{a_3}$ и построим $\beta_2 \vec{a_2}$ и $\beta_3 \vec{a_3}$, где $\beta_2, \beta_3 > 0$. \\
  % Правило параллелограмма
  Т.к. $\vec{a_3}$ лежит на диагонали параллелограмма (из правила сложения векторов параллелограммом), получается, что вектора $\vec{a_1}, \vec{a_2}, \vec{a_3}$ лежат в одной плоскости, что и требовалось доказать.

  (2) Пусть $\vec{a_1}, \vec{a_2}, \vec{a_3}$ лежат в одной плоскости (компланарны). Совместим начала векторов, концы векторов обозначим $A_i$. Проведём через $A_1$ прямую, параллельную $\vec{a_3}$.
  \begin{gather*}
    \overrightarrow{OA'_2} \parallel \overrightarrow{OA_2} \\
    \implies \overrightarrow{OA'_2} = \lambda_2 \overrightarrow{OA_2} \\
    \overrightarrow{OA'_3} \parallel \overrightarrow{OA_3} \\
    \implies \overrightarrow{OA'_3} = \lambda_3 \overrightarrow{OA_3}
  \end{gather*}

  Тогда согласно правилу параллелограмма сложения векторов: \[
    \overrightarrow{OA_1} = \overrightarrow{OA_2} = \overrightarrow{OA_3} \text{, то } \vec{a_1} = \lambda_2 \vec{a_2} + \lambda_3 \vec{a_3}
  \]
\end{proof}

\begin{theorem}
  Любые 4 вектора линейно зависимы.
\end{theorem}

