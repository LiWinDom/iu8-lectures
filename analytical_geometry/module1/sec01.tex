\section{Векторная алгебра}

\begin{definition}
  Вектором называется отрезок, с выбранном на нём направлением.
\end{definition}

\begin{definition}
  Два вектора называется \textbf{коллинеарными}, если они лежат на одной прямой или на параллельных прямых.
\end{definition}

\begin{definition}
Три вектора называются \textbf{компланарными}, если они лежат на прямых, параллельных некоторой плоскости.
\end{definition}

\begin{definition}
  Вектор определяется точкой начала и точкой конца.
  \[
    \overrightarrow{AB}
  .\] 
\end{definition}

Вектор, у которого точка начала фиксирована, называется \textbf{связанным}.

Вектор, у которого точка начала не фиксированная, называется \textbf{свободным}.

Вектор характеризуется \textit{длиной} и \textit{направлением}.

Два вектора называются \textbf{сонаправленными}, если они \textit{коллинеарны} и имеют одно и то же направление.

Два вектора называются \textbf{противоположно направленными} если они \textit{коллинеарны} и имеют противоположные направления.

Два векторы называются равными, если:
\begin{enumerate}
  \item Они коллинеарны и сонаправлены
  \item Их длины равны
\end{enumerate}

\begin{definition}
  Вектор, длина которого равна $1$ называется единичным вектором или \textbf{ортом}.
  \[
    \vec{e} \quad |\vec{e}| = 1 
  .\] 
\end{definition}

\begin{definition}
  Вектор, длина которого равна нулю (начало и конец совпадают) называется \textbf{нулевым вектором}. Направление нулевого вектора произвольное. Нулевой вектор коллинеарен всем векторам.
  \[
    |\vec{0}| = 0
  .\] 
\end{definition}

\begin{definition}
  Суммой векторов $\vec{a}$ и $\vec{b}$ называется $\vec{c}$, который получается по правилу треугольника:
  \begin{enumerate}
    \item Конец вектора $\vec{a}$ совмещают с началом вектора $\vec{b}$
    \item Тогда вектор, идущий из начала вектора $\vec{a}$ к концу вектора $\vec{b}$ и будет вектором $\vec{c}$.
  \end{enumerate}
\end{definition}

\begin{definition}
  Суммой векторов $\vec{a}$ и $\vec{b}$ называется вектор $\vec{c}$, который получается по правилу параллелограмма следующим образом:
  \begin{enumerate}
    \item Совмещают начала векторов $\vec{a}$ и $\vec{b}$
    \item Достраивают фигуры до параллелограмма
    \item Тогда вектор, идущий из начала вектором по диагонали параллограмма и будет исходным вектором $\vec{c}$.
  \end{enumerate}
\end{definition}

\begin{note}
  Если два вектора коллинеарны, то их можно сложить \\ только правилу треугольника.
\end{note}

\begin{definition}
  Произведение вектора $\vec{a}$ на число $\lambda$ называется вектор $\vec{c}$, который будет коллинеарен вектору $\vec{a}$, длина которого будет или меньше в $|\lambda|$ раз и будет сонаправлен, если $\lambda > 0$, и противонаправлен, если $\lambda < 0$.
\end{definition}

\subsection{Свойства векторов}
\begin{gather}
  \vec{a} + \vec{b} = \vec{b} + \vec{a} \\
  (\vec{a} + \vec{b}) + \vec{c} = \vec{a} + (\vec{b} + \vec{c}) \\
  \forall \vec{a} \exists \vec{0} \qquad \vec{a} + \vec{0} = \vec{a} \\
  \forall \vec{a} \exists \vec{b} \qquad \vec{a} + \vec{b} = \vec{0} \implies -\vec{b} = \vec{a} \\
  \lambda\left( \vec{a} + \vec{b} \right) = \lambda \vec{a} + \lambda \vec{b} \\
  \lambda(p \vec{a}) = \left( \lambda p \right) \vec{a} \\
  \left( \lambda  + q\right) \vec{a} = \lambda \vec{a} + q \vec{a}
\end{gather}

\begin{definition}
  Разностью векторов $\vec{a}$ и $\vec{b}$ называется вектор $\vec{c}$, который получается следующим образом:
  \begin{enumerate}
    \item Совмещаем начала вектооров $\vec{a}$ и $\vec{b}$
    \item Вектор, который идёт из конца вектора $\vec{b}$ в начало вектора $\vec{a}$ и есть искомый вектор $\vec{c}$. 
  \end{enumerate}
\end{definition}

\subsection{Ортогональная проекция вектора на направление}

\begin{definition}
  Основание точки $O_a$ перпендикуляра, опущенного их точки $A$ на прямую $L$ называется \textbf{ортогональной проекцией точки} $A$ на прямую $L$.
\end{definition}

\begin{definition}
  Пусть имеем вектор $\overrightarrow{AB}$. Пусть $O_a$ - ортогональная проекция начала вектора $\overrightarrow{AB}$ на прямую $L$, а $O_b$ - это ортогональная проекция конца вектора $\overrightarrow{AB}$ на прямую $L$. Тогда вектор $\overrightarrow{O_aO_b}$, соединяющий проекции и лежащий на прямой $L$, называется \textbf{ортогональной проекцией вектора} $\overrightarrow{AB}$ \textbf{на прямую} $L$.
\end{definition}

\begin{definition}
  \textbf{Осью} называется прямая с выбранным на ней направлением.
\end{definition}

Если на прямой L выбрано направление, то длину $\overrightarrow{O_aO_b}$ берут со знаком $+$, если направление вектора совпадает с выбранным направлением L, и со знаком  $-$, если нет.

\begin{definition}
  Длину вектора $\overrightarrow{O_aO_b}$ со знаком, определяющим \\ направление этого вектора, \textbf{называют ортогональной проекцией вектора} $\overrightarrow{AB}$ \textbf{на ось} $\vec{l}$.
  \[
    np_{\vec{l}}\overrightarrow{AB}
  .\] 
\end{definition}

\begin{definition}
  Ортогональную проекцию вектора на ненулевой вектор $\vec{l}$ называеют \textbf{ортогональной проекцией этого вектора на направление вектора} $\vec{l}$.
\end{definition}

\begin{note}
  Важно! \textit{Ортогональная проекция вектора на направление} - это \textbf{число}!
\end{note}

\begin{theorem}
  Ортогональная проекция вектора $\vec{a}$ на направление ненулевого вектора $\vec{l}$ равна произведению длины вектора $\vec{l}$ на $cos \phi = \widehat{\vec{a}\vec{l}}$
\end{theorem}

\begin{theorem}
  Ортогональная проекция суммы векторов $\vec{a}$ и $\vec{b}$ на направление ненулевого вектора $\vec{l}$ равна сумме ортогональных проекций вектора $\vec{a}$ и $\vec{b}$ на направление ненулевого вектора $\vec{l}$.
  \[
    np_{\vec{l}}\left( \vec{a} + \vec{b} \right) = np_{\vec{l}}\vec{a} + np_{\vec{l}}\vec{b}
  .\] 
\end{theorem}

\begin{theorem}
  Ортогональная проекция вектора произведения $\vec{a}$ и числа $\lambda$ на направление ненулевого вектора $\vec{l}$ равна произведению числа $\lambda$ на ортогональную проекцию вектора $\vec{a}$.
  \[
    np_{\vec{l}} \lambda\vec{a} = \lambda np_{\vec{l}} \vec{a}
  .\] 
\end{theorem}

