\section{Неопределённые интегралы и их свойства}

\begin{definition}
$F(x)$ \textbf{первообразная} для $f(x)$, если $F'(x) = f(x)$ \\
\textit{$f(x), F(x)$ определены на промежутке $I$}
\end{definition}

\begin{theorem}
$F(x)$ --  первообразная для $f(x)$, тогда $F(x) + C$ так же первообразная для $f(x)$
\end{theorem}
\begin{proof}
Пусть $F(x), G(x)$ -- первообразные для $f(x)$ \\
\textit{Рассмотрим $F(x) - G(x)$:}
$$(F(x)- G(x))' = f(x) - f(x) = 0$$
$$F(x) - G(x) = const$$
$$F(x) + C = G(x)$$
\end{proof}

\begin{definition}
\textbf{Интеграл} -- совокупность всех первообразных для $f(x)$ на одном промежутке
$$\int f(x)\,dx = F(x) + C$$
\end{definition}

\begin{note}
Любая непрерывная функция на промежутке имеет первообразную на этом промежутке
\end{note}

\subsection{Свойства}

\begin{itemize}
  \item $(\int f(x)\,dx)' = f(x) \qquad d(\int f(x)\,dx) = f(x)\,dx$
  \item $\int dF(x) = F(x) + C$
  \item $\int af(x)\,dx = a\int f(x)\,dx$
  \item $\int (f(x) + g(x))\,dx = \int f(x)\,dx + \int g(x)\,dx$
\end{itemize}
