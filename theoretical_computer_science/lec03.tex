\subsection{Коды Грея}

\begin{definition}
  Коды, у которых переход к соседнему числу сопровождается изменением только в одном разряде.
\end{definition}

Преимущества:
\begin{itemize}
  \item Упрощение кодирующей логики
  \item Эффективность защиты от нежелательных сбоев
\end{itemize}

Недостатки:
\begin{itemize}
  \item Выполнение арифмитических операций
\end{itemize}

\begin{gather*}
  g_n = b_n \\
  g_i = b_i \xor b_{i+1}, i = \overline{1, n-1}
\end{gather*}
\begin{eg}
  \begin{gather*}
    b = 100101_2 \\
    g = 110111_{g2}
  \end{gather*}
\end{eg}

\begin{table}[htpb]
  \centering
  \caption{Коды Грея}
  \begin{tabular}{ | c | c | c | c |}
  \hline
  0 & 00 & 000 & 0000 \\
  1 & 01 & 001 & 0001 \\
    & 11 & 011 & 0011 \\
    & 10 & 010 & 0010 \\
    &    & 110 & 0110 \\
    &    & 111 & 0111 \\
    &    & 101 & 0101 \\
    &    & 100 & 0100 \\
    &    &     & 1100 \\
    &    &     & 1101 \\
    &    &     & 1111 \\
    &    &     & 1110 \\
    &    &     & 1010 \\
    &    &     & 1011 \\
    &    &     & 1001 \\
    &    &     & 1000 \\
  \hline
  \end{tabular}
\end{table}

\subsubsection{Вычисление двоичного кода по коду Грея}

\begin{gather*}
  b_n = g_n \\
  b_i = \begin{cases}
    g_i \text{, кол-во предшествующих единиц нечётно} \\
    \overline{g_i} \text{, если нет}
  \end{cases}
\end{gather*}

\begin{table}[htpb]
  \centering
  \caption{Перевод кода Грея в двоичную}
  \begin{tabular}{ c }
    11101_{g2} \\
    01110_{g2} \\
    00111_{g2} \\
    00011_{g2} \\
    00001_{g2} \\
    \hline 
    10110_2 \\
  \end{tabular}
\end{table}

\subsection{Трочиная система}

В 1959 в МГУ разработана ЭВМ "Сетунь" на основе троичной системы счисления.

Троичные системы счисления:
\begin{itemize}
  \item Несимметричные: алфавит $\{0, 1, 2\}$ 
  \item Симметричные: алфавит $\{-1, 0, 1\}$ или $\{-, 0, +\}$
\end{itemize}

\subsubsection{Перевод чисел в симметричную троичную систему счисления}

Для перевода из десятичной системы счисления в троичную симметричную систему:
\begin{enumerate}
  \item Делим исходное число на 3
  \item Если остаток от деления равен 0 или 1, то продолжаем процесс деления; если остаток равен 2, то записываем остаток как -, а к частному добавляем 1
  \item Если результат равен 2, то записываем +-
\end{enumerate}

Чтобы число поменяло знак, необходимо все + поменять на - и наоборот (инверсия).
\begin{eg}
  $8 = '+0-_{3ccc}' \to '-8 = -0+_{3ccc}'$
\end{eg}
\begin{eg}
  $261_{10} = '+0+-00_{3ccc}'$ \[
    1 * 3^5 + 0 * 3^4 + 1 * 3^3 - 1 * 3^2 + 0 * 3^1 + 0 * 3^0 = 261
  \]
\end{eg}

\subsubsection{Арифметика в троичной симметричной системе счисления}

\begin{table}[htpb]
  \centering
  \caption{Сложение в 3ссс}
  \label{tab:label}
  \begin{tabular}{ c | c c c }
    + &  + & 0 &  - \\
    \hline
    + & +- & + &  0 \\
    0 &  + & 0 &  - \\
    - &  0 & - & -+ \\
  \end{tabular}
\end{table}

Вычитание осуществляется путём сложения уменьшаемого с инверсией вычитаемого.

\begin{table}[htpb]
  \centering
  \caption{Умножение в 3ссс}
  \label{tab:label}
  \begin{tabular}{ c | c c c }
    * &  + &  0 &  - \\
    \hline
    + &  + &  0 &  - \\
    0 &  0 &  0 &  0 \\
    - &  - &  0 &  + \\
  \end{tabular}
\end{table}

