\section{Непрерывность функции. Точки разрыва}

\begin{definition}
  Функция $f(x)$, определённая в некоторой окрестности точки $x_0$, называется непрерывной в этой точке если: \[
    \exists \lim_{x \to x_0} f(x) = f(x_0)
  \]
\end{definition}

Множество непрерывных функций в точке $x_0$ обозначается $C(x_0)$\[
  f(x) \in C(x_0) \iff \text{ - функция непрерывна в точке } x_0
\] 

\begin{eg}
  \[
  \lim_{x \to 0} \sin(x) = \sin(x) = 0 \iff \sin(x) \in C(0)
  \] 
\end{eg}
\begin{eg}
  \[
  sgn x = \begin{cases}
    1, x > 0 \\
    0, x = 0 \\
    -1, x < 0
  \end{cases} \implies
  sgn \not \in C(0)
  \]
\end{eg}

\begin{definition}
  Функция $y = f(x)$, определённая в некоторой \\ окрестности точки $x_0$, называется непрерывной в этой точке, если в достаточно малой окрестности точки $x_0$ значение функции близки к $f(x_0)$.
  \begin{gather*}
    y = f(x) \in C(x_0) \\
    \iff \\
    (\forall  \varepsilon > 0)(\exists \delta(\varepsilon) > 0)(\forall x \in \mathring{S}(x_0, \delta) \implies |f(x) - f(x_0)| < \varepsilon)
  \end{gather*}
\end{definition}

\begin{definition}
  Функция $y = f(x)$ в некоторой окрестности точки $x_0$ называется непрерывной в этой точке, если выполняются условия:
  \begin{align*}
    &1. \quad \exists \lim_{x \to x_0+} f(x) \\
    &2. \quad \exists \lim_{x \to x_0-} f(x) \\
    &3. \quad \lim_{x \to x_0+} f(x) = \lim_{x \to x_0-} f(x) = f(x) \\
  \end{align*}
\end{definition}

Пусть $y = f(x)$ определена в некоторой точке в окрестности $x_0$.
Выберем произвольный $x$ в этой окрестности.
Тогда:
\begin{align*}
  &\boxed{\Delta x = x - x_0} \text{ - приращение аргумента} \\
  &\boxed{\Delta y = f(x) - f(x_0)} \text{ - соответствующее приращение функции}
\end{align*}

% Добавь наконец шорткаты для align-а!
% и для mathring-а тоже

\begin{definition}
  Функция $y = f(x)$ называется непрерывной в точке $x_0$, если бесконечно малому приращению аргумента соответствует бесконечно малое приращение функции. \[
    \lim_{\Delta x \to 0} \Delta y = 0
  \] 
\end{definition}

\subsection{Односторонняя непрерывность}

\begin{definition}
  Функция $y = f(x)$ определённая в правосторонней окрестности точки $x_0$ (математическим языком - $[x_0, x_0 + \delta)$) называется непрерывной справа в этой точке, если: \[
  \exists \lim_{x \to x_0+} = f(x_0)
  \] 
\end{definition}

\begin{definition}
  Функция $y = f(x)$ определённая в левосторонней окрестности точки $x_0$ (математическим языком - $(x_0 - \delta, x_0]$) называется непрерывной справа в этой точке, если: \[
  \exists \lim_{x \to x_0-} = f(x_0)
  \] 
\end{definition}

\begin{theorem}
  Для того, чтобы функция $y = f(x)$ была непрерывна в точке $x_0$ необходимо и достаточно, чтобы она была непрерывна в этой точке справа и слева. 
\end{theorem}

\begin{definition}
  Функция $y = f(x)$ называется непрерывной на интервале $(a, b)$, если она непрерывна в каждой точке этого интервала.
\end{definition}

\begin{definition}
  Функция $y = f(x) $ называется непрерывной на отрезке $[a, b]$, если:
  \begin{enumerate}
    \item Непрерывна на интервале $(a, b)$
    \item Непрерывна в точке $a$ справа
    \item Непрерывна в точке $b$ слева
  \end{enumerate}
\end{definition}

\begin{itemize}
  \item $C(a, b)$ - множество функций, непрерывных на интервале. 
  \item $C[a, b]$ - множество функций, непрерывных на отрезке. 
  \item $C(X)$ - множество функций, непрерывных на промежутке $X$. 
\end{itemize}

\subsection{Классификация точек разрыва}

\begin{definition}
  Пусть функция $y = f(x)$ определена в некоторой точке проколотой окрестности точки $x_0$ непрерывна в любой точке этой окрестности (за исключением самой точки $x_0$).
  Тогда точка $x_0$ называется точкой разрыва функции.
\end{definition}

% Переверстать
Пусть точка $x_0$ - точка разрыва. Её можно классифицировать как:
\begin{itemize}
  \item I-ого рода
    \begin{itemize}
    \item Основное условие \[
      \exists \lim_{x \to x_0 +-}
\]
    \item Точка конечного разрыва \[
      \lim_{x \to x_0+} \neq \lim_{x \to x_0-}   
    \] 
    \item Точка устранимого разрыва \[
      \lim_{x \to x_0+} = \lim_{x \to x_0-} \neq f(x_0) \text{ или } \not\exists f(x_0)   
    \]  
  \end{itemize}
  \item II рода \[
  \not \exists \lim_{x \to x_0+-} 
  \] 
\end{itemize}

\begin{definition}
  Если точка $x_0$ -- точка разрыва функции $y = f(x)$ и существуют конечные пределы $\lim_{x \to x_0+} f(x)$ и $\lim_{x \to x_0-} f(x)$, то $x_0$ называют точкой I-го рода.
\end{definition}

\begin{definition}
  Если точка $x_0$ -- точка разрыва функции $y = f(x)$ и \textbf{не} существуют конечные пределы $\lim_{x \to x_0+} f(x)$ и $\lim_{x \to x_0-} f(x)$ или \\ $\lim_{x \to x_0} f(x) = \infty$, то $x_0$ называется точкой разрыва II-го рода.
\end{definition}

\begin{definition}
  Если точка $x_0$ -- точка разрыва первого рода функции $y = f(x)$, и предел $\lim_{x \to x0+} f(x) \neq \lim_{x \to x_0-} f(x)$, то $x_0$ называется точкой конечного разрыва или точкой \textit{скачка}.
\end{definition}

\begin{definition}
  Если точка $x_0$ -- точка разрыва первого рода функции $y = f(x)$, и предел  $\lim_{x \to x_0+} f(x) = \lim_{x \to x_0-} f(x)$, но $\neq f(x_0)$, то точка $x_0$ называется точкой устранимого разрыва.
\end{definition}

\subsubsection*{Примеры}

\begin{eg}
  \begin{gather*}
    y = \frac{|x - 1|}{x - 1} \\
    D_f = \R \setminus \{1\} \\
    x = 1 \text{ - точка разрыва} \\
  \lim_{x \to 1+} f(x) = \lim_{x \to 1+} \frac{|x - 1|}{x - 1} = \frac{x - 1}{x - 1} = 1 \\
  \lim_{x \to 1-} f(x) = \lim_{x \to 1-} \frac{|x - 1|}{x - 1} = \frac{1 - x}{x - 1} = -1 \\
  \lim_{x \to 1+} f(x) \neq \lim_{x \to 1-} f(x) \\
  \implies x = 1 \text{ - т.р. I рода, точка скачка}\\
  \Delta f = |\lim_{x \to 1+} f(x) - \lim_{x \to 1-} f(x)| = |1 - (-1)| = 2  
  \end{gather*}
\end{eg}

% Добавить шорткаты для Df, Dx, Dy

\begin{eg}
  \begin{gather*}
    y = \frac{\sin(x)}{x} \\
    D_f = \R \setminus \{0\} \\
    \lim_{x \to 0+} f(x) = \lim_{x \to 0+} \frac{\sin(x)}{x} = 1 \\
    \lim_{x \to 0-} f(x) = \lim_{x \to 0-} \frac{\sin(x)}{x} = 1 \\
    \lim_{x \to 0+} f(x) = \lim_{x \to 0-} f(x) \\
    \implies x = 0 - \text{ т.р. I рода, устранимая точка разрыва} \\
    g(x) = \begin{cases}
      \frac{\sin(x)}{x}, x \neq 0 \\
      1, x = 0
    \end{cases} \\
    f(x) \not\in C(0) \\
    g(x) \in C(0)
  \end{gather*}
\end{eg}

\begin{eg}
  \begin{gather*}
    y = e^{\frac{1}{x}} \\
    D_f = \R \setminus \{0\} \\
    \lim_{x \to 0+} f(x) = \lim_{x \to 0+} e^{\frac{1}{x}} = e^{+\infty} = \infty \\ 
    \lim_{x \to 0-} f(x) = \lim_{x \to 0-} e^{\frac{1}{x}} = e^{-\infty} = 0 \\ 
    \lim_{x \to 0+} f(x) = \infty \\
    \implies x = 0 \text{ - т.р. II рода}
  \end{gather*}
\end{eg}

\subsection{Свойства непрерывных функций в точке}

\begin{theorem}
  Пусть функции: \[
    \begin{rcases*}
      y = f(x) \\
      y = g(x)
    \end{rcases*}
    \in C(x_0)
  \] 
  Тогда:
  \begin{gather*}
    f(x) + g(x) \in C(x_0) \\
    (f \cdot g)(x) \in C(x_0)
  \end{gather*}
\end{theorem}
\begin{proof}
  По определению непрерывной функции: 
  \begin{gather*}
    \lim_{x \to x_0} f(x) = f(x_0) \\
    \lim_{x \to x_0} g(x) = g(x_0) \\
  \end{gather*}
  Рассмотрим:
  \begin{gather*}
    \lim_{x \to x_0} (f(x) + g(x)) = \lim_{x \to x_0} f(x) + \lim_{x \to x_0} g(x) + f(x_0) = g(x_0) \\
    \implies f(x) + g(x) \in C(x_0) 
    \\
    \lim_{x \to x_0} (f \cdot g)(x) = \lim_{x \to x_0} f(x) \cdot g(x) = \lim_{x \to x_0} f(x) \cdot \lim_{x \to x_0} g(x) = f(x_0) \cdot g(x_0) 
    \\
    \implies (f \cdot g)(x) \in C(x_0)
    \\
    \lim_{x \to 0} \frac{f(x)}{g(x)} = \frac{\lim_{x \to x_0} f(x)}{\lim_{x \to x_0} g(x_0)}
  \end{gather*}
\end{proof}

\begin{theorem}
  Пусть \[
    g(y) \in C(y_0), \quad y_0 = \lim_{x \to x_0} f(x)
  \]
  Тогда: \[
    \lim_{x \to 0} g(f(x)) = g(\lim_{x \to 0} f(x))
  \]
\end{theorem}
\begin{proof}
  Т.к. функция $g\left( y \right) \in C(y_0)$, то $\lim_{y \to y_0} g(y) = g(y_0)$.
  С другой стороны, по условию $\lim_{x \to x_0} f(x) = y_0$.
  По теореме \textquote{О пределе сложной функции} $\exists \lim_{x \to x_0} g(f(x))$.
  Подставим в последнее равенство $y_0 = \lim_{x \to x_0} f(x)$: \[
    \lim_{x \to x_0} g(f(x)) = g(\lim_{x \to x_0} f(x))
  \]  
\end{proof}

\begin{theorem}
 \textit{О непрерывности сложной функции}. \\
  Пусть функция $y = f(x) \in C(x_0)$, а функция $g(y_0) \in C(y_0)$, причем $y_0 = f(x_0)$.
  Тогда сложная функция $F(x) = g(f(x)) \in C(x_0)$.
\end{theorem}
\begin{proof}
  Т.к. $y = f(x) \in C(x_0)$, то по определению непрерывности $\implies \lim_{x \to x_0} f(x) = f(x_0)$. 
  Аналогично для $f(x) \in C(x_0)$ -- по определению непрерывности $\implies \lim_{y \to y_0} g(y) = g(y_0)$. 
  Рассмотрим $\lim_{x \to x_0} F(x) = \lim_{x \to x_0} g(f(x))$.
  По теореме 29: \[
    \lim_{x \to x_0} g(f(x)) = g(\lim_{x \to x_0} f(x)) = 
  \] 
  По непрерывности функции: \[
    = g(f(x_0)) = F(x_0) \implies g(f(x)) \in C(x_0)
  \] 
\end{proof}

\begin{theorem}
  \textit{О сохранении знака непрерывной функции в окрестности точки}. \\
  Если функция $f(x) \in C(x_0)$ и $f(x_0) \neq 0$, то $\exists S(x_0)$, в которой знак значения функции совпадает со знаком $f(x_0)$.
\end{theorem}
\begin{proof}
  Т.к. функция $y = f(x) \in C(x_0)$, то $\lim_{x \to x_0} f(x) = f(x_0)$. 
  По теореме о сохранении функции знака своего предела $\implies \exists S(x_0)$, в которой знак значений функции совпадает со знаком $f(x_0)$.
\end{proof}
\begin{note}
  На экзамене требуется доказать также и теорему о сохранении функции знака своего предела!
\end{note}

\subsection{Непрерывность элементарных функций}

\begin{theorem}
  \textit{Основные элементарные функции непрерывные в области определения}. \\
\end{theorem}
\begin{proof}
  Это теорема доказывается для каждой элементарной функции отдельно.
  Докажем её для функций $y = \sin(x), y = \cos(x)$:
    \begin{gather*}
      y = \sin(x), D_y = \R \\
      x_0 = 0, \lim_{x \to x_0} \sin(x) = \sin(0) \implies y = \sin(x) \in C(0) \\
      \forall x \in D_y= \R, \quad \Delta x \text{ -- приращение функции} \\ 
      x = x_0 + \Delta x, \quad x \in D_f = \R \\
      \Delta y = y(x) - y(x_0) = y(x_0 + \Delta x) - y(x_0) \\
      = \sin(x_0 + \Delta x) - \sin(x_0) = 2\sin\left( \frac{x_0 + \Delta x - x_0}{2} \right) \cos\left( \frac{x_0 + \Delta x + x_0}{2} \right) \\
        = 2\sin\left( \frac{\Delta x}{2} \right) \cos\left(x_0 + \frac{\Delta x}{2} \right) \\
      \lim_{\Delta x \to 0} \Delta y = \lim_{\Delta x \to 0} 2\sin\left( \frac{\Delta x}{2} \right) \cos\left(x_0 + \frac{\Delta x}{2}\right) = 0 \\
        \text{ -- по т. об произв. огр. на б.м.ф.}
    \end{gather*}
    Т.к. $\lim_{\Delta x \to 0} \Delta y = 0$ по опр. непр. функции $\implies y =\sin(x)$ непрерывна в точке $x_0$. 
    Т.к. $x_0$ -- произвольная точка из области определения, то $y = \sin(x)$ непрерывна на всей области произведения.
\end{proof}

\begin{theorem}
  \textit{Элементарные функции непрерывны в области определения} \\
\end{theorem}
\begin{proof}
  Доказательство данной теоремы следует из определения элементарных функций с помощью операций \textit{сложения, вычитания, умножения, композиции}, предыдущей теоремы, теоремы об алгебраических свойствах непрерывной функции и теоремы о композиции непрерывных функций. 
\end{proof}

\subsection{Свойства функций, непрерывных на промежутке}

\begin{theorem}
  \textit{Об ограниченности непрерывной функции} или \textit{Первая теорема Вейерштрасса}. . \\
  Если функция $y = f(x)$ непрерывна на отрезке $ab$, то она на этом отрезке ограниченна. \[
    f(x) \in C[a, b] \implies \exists M \in \R, M > 0, \forall x \in [a, b] : |f(x)| \le  M
  \] 
\end{theorem}

\begin{theorem}
  \textit{О достижении непрерывной функции наибольшего и наименьшего значений} или \textit{Вторая теорема Вейерштрасса}. \\
  Если функция $y = f(x) \in C[a, b]$, то она достигает на этом отрезке своего наибольшего и наименьшего значения.
  \begin{gather*}
    f(x) \in C[a, b] \\
    \implies \\
    \exists x_*, x^* \in [a, b] : \forall x \in [a, b] \implies m = f(x_*) \le f(x) \le f(x^*) = M
  \end{gather*}
\end{theorem}

\begin{theorem}
  \textit{О существовании нуля непрерывной функции} или \textit{Первая теорема Бальцана-Коши}. \\
  Если функция $y = f(x) \in C[a, b]$, и на концах отрезка принимает значения разных знаков, то $\exists c \in (a, b) : f(c) = 0$. \[
    f(x) \in S[a, b] \land f(a) \cdot f(b) < 0 \implies \exists  c \in (a, b) : f(c) = 0
  \] 
\end{theorem}

\begin{theorem}
  \textit{О промежуточном значении непрерывной функции} или \textit{Вторая теорема Бальцана-Коши}. \\
  Если функция $y = f(x) \in C[a, b]$ и принимает на границах отрезка различные значения $f(a) = A \neq f(b) = B$, то $\forall C \in [A, B] \exists c \in (a, b)$, в которой $f(c) = C$.  \[
    f(x) \in C[a, b] \land f(a) = A \neq f(b) = B \\
    \implies \\
    \exists C \in (A, B) \implies \exists c \in (a, b) : f(c) = C
  \] 
\end{theorem}

\begin{theorem}
  \textit{О существовании обратной к непрерывной функции}. \\
  Пусть $y = f(x) \in C(a, b)$ и строго монотонна на этом интервале. Тогда в соответствующем $(a, b)$ интервале значений функции существует обратная функция $x = f^{-1}(y)$, которая так же строго монотонна и непрерывна.
\end{theorem}

