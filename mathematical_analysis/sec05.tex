\section{Бесконечно малые функции}

\begin{definition}
  Функция называется \textbf{бесконечно малой} при $x \to x_0$, если предел функции в этой точке равен $0$. Кратко - \textbf{б.м.ф.} или \textbf{б.м.в}.
  \begin{gather*}
    \lim_{x \to x_0} f(x) = 0 \\
    (\forall \varepsilon > 0)(\exists \delta(\varepsilon)) (\forall x \in \mathring{S}(x_0, \delta) \implies |f(x)| < \varepsilon )
  \end{gather*}
\end{definition}
\begin{note}
  Стремление аргумента может быть \textit{любое}, главное, чтобы предел был равен нулю.
\end{note}

Бесконечно малые функции обозначаются $\alpha(x), \beta(x), \gamma(x)\ldots$

\begin{eg}
  \begin{gather*}
    y = x - 2 \\
    \lim_{x \to 2} (x - 2) = 0 \\ 
  \end{gather*}
  $y = x - 2$ при $x \to 2$ является бесконечно малой.
\end{eg}
\begin{eg}
  \begin{gather*}
    y = \sin(x) \\
    \lim_{x \to 0} \sin(x) = 0
  \end{gather*}
  $y = \sin(x)$ при $x \to 0$ является бесконечно малой.
\end{eg}
\begin{eg}
  \begin{gather*}
    y = \sin(\frac{1}{x}) \\
    \lim_{x \to \infty} \sin(\frac{1}{x}) = 0
  \end{gather*}
  $y = \sin(\frac{1}{x})$ при $x \to \infty$ является бесконечно малой.
\end{eg}

\subsection{Свойства бесконечно малых функций}

\begin{theorem}
  \textit{О сумме конечного числа бесконечно малых функций}. \\
  Конечная сумма бесконечно малых функции есть бесконечно малая функция.
\end{theorem}
\begin{proof}
  Пусть дано конечное число бесконечно малых функций, например две: $\alpha(x), \beta(x)$.
  Тогда по определению бесконечно малой функции:
  \[
  \lim_{x \to x_0} \alpha(x) = 0 \qquad \lim_{x \to x_0} \beta(x) = 0
  \] 
  Нужно доказать, что:
  \[
  \lim_{x \to x_0} (\alpha(x) + \beta(x)) = 0
  \] 
  Распишем:
  \begin{gather*}
    \lim_{x \to x_0} \alpha(x) = 0 \iff \\
    (\forall \varepsilon_1 = \frac{\varepsilon}{2} > 0)(\exists \delta_1 > 0)(\forall x \in \mathring{S}(x_0, \delta_1) \implies |\alpha(x)| < \frac{\varepsilon}{2}) \tag{1}\\
    \lim_{x \to x_0} \beta(x) = 0 \iff \\
    (\forall \varepsilon_2 = \frac{\varepsilon}{2} > 0)(\exists \delta_2 > 0)(\forall x \in \mathring{S}(x_0, \delta_2) \implies |\beta(x)| < \frac{\varepsilon}{2}) \tag{2}
  \end{gather*}

  Выберем $\delta = min \{\delta_1, \delta_2\}$. Тогда (1) и (2) верны одновременно. Получаем:
  \begin{gather*}
    (\forall \varepsilon > 0)(\exists \delta > 0)(\forall x \in \mathring{S}(x_0, \delta) \\ \implies |\alpha(x) + \beta(x)| \le |\alpha(x)| + |\beta(x)| < \frac{\varepsilon}{2} + \frac{\varepsilon}{2} = \varepsilon) 
  \end{gather*}
  Тогда по определнию бесконечно малой функции:
  \[
  \lim_{x \to x_0} (\alpha(x) + \beta(x)) = 0
  \] 
\end{proof}

\begin{theorem}
  \textit{О произведении бесконечно малой функций на локально ограниченную}. \\
  Произведение бесконечно малой функции на локальной ограниченную есть величина бесконечно малая.
\end{theorem}
\begin{proof}
  Пусть $\alpha(x)$ - бесконечно малая функция при $x \to x_0$, а функция $f(x)$ при $x \to  x_0$ является локально ограниченной. Доказываем, что: \[
  \alpha(x) \cdot f(x) = 0
  \] 
  Распишем:
  \begin{gather*}
    \lim_{x \to x_0} \alpha(x) = 0 \\
    \iff (\forall \varepsilon_1 = \frac{\varepsilon}{M} > 0)(\exists \delta_1 > 0)(\forall x \in \mathring{S}(x_0, \delta_1) \implies |\alpha(x)| < \varepsilon_1 = \frac{\varepsilon}{M}) \tag{1}\\
    M \in \R, M > 0 \\
    \forall x \in  \mathring{S}(x_0, \delta_2) \implies |f(x)| < M \tag{2} \\ 
  \end{gather*}

  Выберем $\delta = min \{\delta_1, \delta_2\} $, тогда (1) и (2) верны одновременно. В итоге получаем:
  \begin{gather*}
    (\forall \varepsilon > 0)(\exists \delta > 0)(\forall x \in \mathring{S}(x_0, \delta) \implies \\
    |\alpha(x) \cdot f(x)| = |\alpha(x)| \cdot |f(x)| < \frac{\varepsilon}{M} \cdot M < \varepsilon  
  \end{gather*}

  Тогда по определению бесконечно малой функции:
  \[
  \lim_{x \to x_0} \alpha(x) \cdot f(x) = 0
  \] 
\end{proof}
\begin{eg}
  \begin{gather*}
    \lim_{x \to \infty} \frac{\sin(x)}{x} = \lim_{x \to \infty} \frac{1}{x} \cdot \sin(x) = 0
  \end{gather*}
  Т.к. $\sin(x)$, при $x \to \infty$ является локально ограниченной $\sin(x) \le 1$.
\end{eg}

\begin{theorem}
  \textit{О связи функции, её предела и бесконечно малой}. \\
  Функция $y = f(x)$ имеет конечный предел в точке  $x_0$ тогда и только тогда, когда её можно представить в виде суммы предела и некоторой бесконечно малой функции.
  \begin{gather*}
    \lim_{x \to x_0} f(x) = a \iff f(x) = a + \alpha(x), \text{где } \alpha(x) - \text{б.м.ф при } x \to x_0
  \end{gather*}
\end{theorem}
\begin{proof}
  \textit{Необходимость.} \\
  Дано: \[
    \lim_{x \to x_0} f(x) = a
  \]
  Доказать: \[
    f(x) = a + \alpha(x), \text{где } \alpha(x) \text{ - б.м.ф. при } x \to  x_0
  \]
  Распишем: \[
    \lim_{x \to x_0} f(x) = a \iff (\forall \varepsilon > 0)(\exists \delta > 0)(\forall x \in \mathring{S}(x_0, \delta) \implies |f(x) - a| < \varepsilon)  
  \]
  Обозначим $f(x) - a = \alpha(x)$, тогда: \[
    \lim_{x \to x_0} f(x) = a \iff (\forall \varepsilon > 0)(\exists \delta > 0)(\forall x \in \mathring{S}(x_0, \delta) \implies |\alpha(x)| < \varepsilon)  
  \]
  По определению бесконечно малой функции $\alpha(x)$ - бесконечно малая функция. Из обозначения следует, что: \[
    f(x) = a + \alpha(x)
  \]
  где $\alpha(x)$ - бесконечно малая функция при $x \to x_0$.

  \textit{Достаточность.} \\
  Дано: \[
    f(x) = a + \alpha(x), \text{где } \alpha(x) \text{ - б.м.ф. при } x \to x_0
  \]
  Доказать: \[
    \lim_{x \to x_0} f(x) = a
  \]
  По определению б.м.ф.: \[
    \lim_{x \to x_0} \alpha(x) = 0 \iff (\forall \varepsilon > 0)(\exists \delta > 0)(\mathring{S}(x_0, \delta) \implies |\alpha(x)| < \varepsilon 
  \]
  С учётом введённого обозначения: \[
    (\forall \varepsilon > 0)(\exists \delta > 0)(\mathring{S}(x_0, \delta) \implies |f(x) - a| < \varepsilon \iff \lim_{x \to x_0} f(x) = a
  \]
\end{proof}
\begin{corollary}
  Т.к. любая бесконечно малая функция локально ограничена, то произведение двух бесконечно малых функций есть бесконечно малая функция.
\end{corollary}
\begin{corollary}
  Произведение бесконечно малой функции на константу \\ есть величина бесконечно малая.
\end{corollary}

\section{Арифметические операции над функциями, имеющими конечный предел}

Пусть $f(x)$ и  $g(x)$ имеют конечные пределы в точке $x_0$.

\begin{theorem}
  Предел суммы (разности) двух функций, имеющих конечные пределы равен сумме (разности) пределов.\[
  \lim_{x \to x_0} (f(x) \pm g(x)) = \lim_{x \to x_0} f(x) + \lim_{x \to x_0} g(x)
  \] 
\end{theorem}

\begin{theorem}
  \textit{О пределе отношения функций}. \\ 
  Предел отношения двух функций, имеющих конечный предел, равен частному их пределов при условии, что предел в знаменателе отличен от нуля. \[
  \lim_{x \to x_0} \left(\frac{f(x)}{g(x)} \right) = \frac{\lim_{x \to x_0} f(x)}{\lim_{x \to x_0} g(x)}, \lim_{x \to x_0} g(x) \neq 0
  \] 
\end{theorem}

\begin{theorem}
  \textit{О пределе произведения функций}. \\
  Предел произведения функций равен произведению пределов.
  \begin{gather*}
    \lim_{x \to x_0} (f(x) \cdot g(x)) = \lim_{x \to x_0} f(x) \cdot \lim_{x \to x_0} g(x)
  \end{gather*}
\end{theorem}
\begin{proof}
  Пусть:
  \begin{gather*}
    \lim_{x \to x_0} f(x) = a \tag{1} \\
    \lim_{x \to x_0} f(x) = b \tag{2} \\
  \end{gather*}

  По теореме о связи функции, её предела и бесконечно малой функции:
  \begin{gather*}
    (1) \implies f(x) = a + \alpha(x) \text{, где } \alpha(x) \text{ - б.м.ф.} \\
    (2) \implies f(x) = b + \beta(x) \text{, где } \beta(x) \text{ - б.м.ф.}
  \end{gather*}

  Рассмотрим:
  \begin{gather*}
    \begin{align*}
      f(x) \cdot g(x) &= (a + \alpha(x))(b + \beta(x)) \\
                      &= ab + \underbrace{a \cdot \beta(x) + b \alpha (x) + \alpha(x) \cdot \beta(x)}_{\gamma(x)} \\
                  &= ab + \gamma(x) \\
    \end{align*}
  \end{gather*}

  По следствию из теоремы 15:
  \begin{gather*}
    a \cdot \beta(x) = \text{б.м.ф. при } x \to 0 \\ 
    b \cdot \alpha(x) = \text{б.м.ф. при } x \to 0 \\ 
    \alpha(x) \cdot \beta(x) = \text{б.м.ф. при } x \to 0 \\ 
  \end{gather*}

  По теореме о сумме конечного числа с б.м.ф.:
  \begin{gather*}
    \gamma(x) = \text{б.м.ф. при } x \to 0 \\ 
  \end{gather*}

  Далее расписываем предел:
  \begin{gather*}
    \begin{align*}
      \lim_{x \to x_0} f(x) \cdot g(x) &= \lim_{x \to x_0} (f(x) \cdot g(x)) \\
                                &= \lim_{x \to x_0} ab + \lim_{x \to x_0} \gamma(x) \\
                                &= ab + 0 \\
                                &= ab \\
    \end{align*}
  \end{gather*}
\end{proof}
\begin{corollary}
  \begin{gather*}
    \lim_{x \to x_0} (c \cdot  f(x)) = c \cdot \lim_{x \to x_0} f(x)
  \end{gather*}  
\end{corollary}

