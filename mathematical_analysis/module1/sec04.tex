\subsection{Основные теоремы о пределах}

\begin{theorem}
  \textit{О локальной ограниченности функции, имеющей конечный предел}. \\
  Функция, имеющая конечный предел, локально ограничена.
\end{theorem}
\begin{proof}
  \begin{gather*}
    \lim_{x \to x_0} f(x) = a \\
    \iff (\forall \varepsilon > 0)(\exists \delta(\varepsilon) > 0) (\forall x \in \mathring{S}(x_0, \delta) \implies |f(x) - a| < \varepsilon) \\
  \end{gather*}
  Распишем:
  \begin{gather*}
    \begin{matrix}
      - \varepsilon < f(x) - a < \varepsilon \\
      a - \varepsilon < f(x) < a + \varepsilon \\
    \end{matrix}
    \qquad
    \forall  x \in \mathring{S}(x_0, \delta)
  \end{gather*}
  Выберем $M = max\{|a - \varepsilon|, |a + \varepsilon|\}$ 
  \begin{gather*}
    |f(x)| \le  M, \quad \forall  x \in  \mathring{S}(x_0, a)
  \end{gather*}
  Что и требовалось доказать.
\end{proof}

\begin{theorem}
  \textit{О единственности предела функции}. \\
  Если функция имеет конечный предел, то он единственный.
\end{theorem}
\begin{proof}
 Предположим, что функция имеет более одного предела, например 2 - $a$ и $b$. Тогда:
 \begin{gather*}
   \lim_{x \to x_0} = a \tag{1} \\  
   \lim_{x \to x_0} = b  \tag{2} \\
    a \neq b \text{, пусть } b > a
 \end{gather*}
 \begin{gather*}
   (1) \iff (\forall  \varepsilon_1 > 0)(\exists  \delta_1(\varepsilon_1) > 0)(\forall  x \in  \mathring{S}(x_0, \delta_1) \implies |f(x) - a| < \varepsilon_1) \\
   (2) \iff (\forall  \varepsilon_2 > 0)(\exists  \delta_2(\varepsilon_2) > 0)(\forall  x \in  \mathring{S}(x_0, \delta_2) \implies |f(x) - b| < \varepsilon_2) \\
 \end{gather*}
 Распишем:
 \begin{gather*}
   (1) \implies a - \varepsilon_1 < f(x) < a + \varepsilon_1, \forall  x \in \mathring{S}(x_0, \delta_1) \\
   (2) \implies b - \varepsilon_2 < f(x) < b + \varepsilon_2, \forall  x \in \mathring{S}(x_0, \delta_2) \\
 \end{gather*}
 Выберем $\delta = min \{\delta_1, \delta_2\}$, тогда $\forall x \in  \mathring{S}(x_0, \delta)$
 будет верно (1) и (2) одновременно.

 Пусть $\varepsilon_1 = \varepsilon_2 = \varepsilon = \frac{b - a}{2}$:

 \begin{gather*}
  (1) \implies f(x) < a + \varepsilon_1 = a + \frac{b - a}{2} = \frac{a + b}{2} \\
  (2) \implies f(x) > b - \varepsilon_2 = b - \frac{b - a}{2} = \frac{a + b}{2} \\
  \forall x \in \mathring{S}(x_0, \delta)
 \end{gather*}
 
 Мы получили противоречие. Это означает, что предположение не является верным. Функция имеет единственный предел.
\end{proof}

\begin{theorem}
  \textit{О сохранении функией знака своего предела} \\
  Если $\lim_{x \to x_0} = a \neq 0$, то $\exists \mathring{S}(x_0, \delta)$ такая, что функция в ней сохраняет знак своего предела. \[
  \lim_{x \to x_0} f(x) = a \neq 0 \to 
  \begin{matrix}
    a > 0 \\
    a < 0
  \end{matrix}
  \implies 
  \begin{matrix}
    f(x) > 0 \\
    f(x) < 0
  \end{matrix}
  \quad
  \forall x \in \mathring{S}(x_0, \delta)
  \] 
\end{theorem}
\begin{proof}
  Пусть $a > 0$. Выберем  $\varepsilon = a > 0$.
  \begin{gather*}
    \lim_{x \to x_0} = a \iff (\forall \varepsilon = a)(\exists  \delta(x) > 0) (\forall x \in \mathring{S}(x_0, \delta) \implies \\
    |f(x)- a| < \varepsilon = a) 
  \end{gather*}

  Распишем:
  \begin{gather*}
    -a < f(x) - a < a \\
    \boxed{0 < f(x) < 2a}
  \end{gather*}
  Знак у функции $f(x)$ и числа $a$ - одинаковые.

  Пусть $a < 0$. Выберем  $\varepsilon = -a$.
  \begin{gather*}
    \lim_{x \to x_0} f(x) = a \iff (\forall \varepsilon = -a)(\exists  \delta(x) > 0) (\forall x \in \mathring{S}(x_0, \delta) \implies \\
    |f(x) - a| < \varepsilon = -a) 
  \end{gather*}

  Распишем:
  \begin{gather*}
    -a < f(x) - a < a \\
    \boxed{-2a < f(x) < 0}
  \end{gather*}
  Знак у функции $f(x)$ и числа  $a$ - одинаковые.

  Значит, $f(x)$ сохраняет знак своего предела  $\forall x \in \mathring{S}(x_0, \delta)$ 
\end{proof}

\begin{corollary}
  Если функция $y = f(x)$ имеет предел в точке  $x_0$ и знакопостояна в $\mathring{S}(x_0, \delta)$, тогда её предел не может иметь с ней противоположные знак.
\end{corollary}

\begin{theorem}
  \textit{О предельном переходе в неравенстве}. \\
  Пусть существуют конечные пределы функций $f(x)$ и  $g(x)$ в точке $x_0$ и $\forall x \in \mathring{S}(x_0, \delta)$ верно $f(x) < g(x)$. Тогда $\forall x \in \mathring{S}(x_0, \delta)$ имеет место неравенство $\lim_{x \to x_0} f(x) \le \lim_{x \to x_0} g(x)$.
\end{theorem}
\begin{proof}
  По условию $f(x) < g(x), \forall x \in \mathring{S}(x_0, \delta)$. \\
  Введём функцию $F(x) = f(x) - g(x) < 0, \forall x \in \mathring{S}(x_0, \delta)$. 
  Т.к. $f(x)$ и $g(x)$ имеют конечные пределы в точке $x_0$, соответственно и функция $F(X)$ имеет конечный предел в точке $x_0$ (как разность $f(x)$ и $g(x)$).
  
  По следствию из предыдущей теоремы
  $\implies \lim_{x \to x_0} F(x) $ 
  
  Подставим $F(x) = f(x) - g(x)$: \[
    \lim_{x \to x_0} \left( f(x) - g(x) \right) \le 0 \implies \lim_{x \to x_0} f(x) - \lim_{x \to x_0} g(x) \le 0 \implies \\
    \lim_{x \to x_0} f(x) \le \lim_{x \to x_0} g(x) 
  \]
\end{proof}
\begin{eg}
  Пусть $f(x) = 0$, $g(x) = x^2$ и $x_0 = 0$. \\
  \begin{gather*}
    \forall x \in \mathring{S}(x_0, \delta) \qquad 0 < x^2 \\
    \lim_{x \to 0} f(x) = 0 \\
    \lim_{x \to 0} g(x) = \lim_{x \to 0} x^2 = 0 \\
    \lim_{x \to 0} f(x) = \lim_{x \to 0} g(x) \\
    \lim_{x \to 0} f(x) \le \lim_{x \to 0} g(x) 
  \end{gather*}
  В теореме знак \textbf{строгий} переходит в \textbf{нестрогий}!  
\end{eg}

\begin{theorem}
  \textit{О пределе промежуточной функции.} \\
  Пусть существуют конечные пределы функций $f(x)$ и $g(x)$ в точке  $x_0$ и $\lim_{x \to x_0} f(x) = a$ и $\lim_{x \to x_0} g(x) = a$, $\forall x \in \mathring{S}(x_0, \delta)$ верно неравенство $f(x) \le h(x) \le g(x)$. Тогда $\lim_{x \to x_0} h(x) = a$.
\end{theorem}
\begin{proof}
  По условию: 
  \begin{gather*}
    \lim_{x \to x_0} f(x) = a \iff (\forall \varepsilon > 0)(\exists \delta_1(\varepsilon) > 0)(\forall x \in \mathring{S}(x_0, \delta) \implies |f(x) - a| < \varepsilon) \tag{1} \\
    \lim_{x \to x_0} g(x) = a \iff (\forall \varepsilon > 0)(\exists \delta_2(\varepsilon) > 0)(\forall x \in \mathring{S}(x_0, \delta) \implies |g(x) - a| < \varepsilon) \tag{2}
  \end{gather*}
  Выберем $\delta_0 = min \{\delta, \delta_1, \delta_2\}$, тогда (1), (2) и $f(x) \le h(x) \le g(x)$ верны одновременно $\forall x \in \mathring{S}(x_0, \delta_0)$.
  \begin{align*}
    (1) \quad a - \varepsilon < f(x) < a + \varepsilon \\
    (2) \quad a - \varepsilon < g(x) < a + \varepsilon \\
  \end{align*}
  \begin{gather*}
    f(x) \le h(x) \le g(x) \\
    \implies a - \varepsilon_1 < f(x) \le h(x) \le g(x) < a + \varepsilon_2 \\
    \implies \forall x \in \mathring{S}(x_0, \delta_0) \qquad a - \varepsilon < h(x) < a + \varepsilon
  \end{gather*}
  В итоге:
  \begin{gather*}
    (\forall \varepsilon > 0)(\exists \delta_0(\varepsilon) > 0)(\forall x \in \mathring{S}(x_0, \delta_0 \implies |h(x) - a| < \varepsilon) \\
    \implies \text{по определению предела} \\
    \lim_{x \to x_0} h(x) = a
  \end{gather*}
\end{proof}

\begin{theorem}
  \textit{О пределе сложной функции}. \\
  Если функция $y = f(x)$ имеет предел в точке  $x_0$ равный $a$, то функция  $\varphi(y)$ имеет предел в точке $a$, равный $C$, тода сложная функция  $\varphi(f(x))$ имеет предел в точке $x_0$, равный $C$.
  \begin{gather*}
    \begin{rcases}
      y = f(x) \\
      \lim_{x \to x_0} f(x) = a \\
      \lim_{y \to a} \varphi(y) = C \\
    \end{rcases}
    \implies \lim_{x \to x_0} \varphi(f(x)) = C
  \end{gather*}
\end{theorem}
\begin{proof}
  \begin{gather*}
    \lim_{y \to a} \varphi(y) \iff (\forall \varepsilon > 0)(\exists \delta_1 > 0)(\forall y \in \mathring{S}(a, \delta_1) \implies |\varphi(y) - a| < \varepsilon) \tag{1}
  \end{gather*}
  Выберем в качестве $\varepsilon$ в пределе найденное $\delta_1$: \[
    \lim_{x \to x_0} f(x) = a \iff (\forall \delta_1 > 0)(\exists \delta_2 > 0)(\forall x: 0 < |x - x_0| < \delta_2 \implies |f(x) - a| < \delta_1 \tag{2} 
  \] 
  В итоге: \[
  (\forall \varepsilon > 0)(\exists \delta_2 > 0)(\forall x: 0 < |x - x_0| < \delta_2 \implies |\varphi(f(x)) - c| < \varepsilon)
  \] 
  Что равносильно: \[
  \lim_{x \to x_0} \varphi(f(x)) = c
  \] 
\end{proof}

