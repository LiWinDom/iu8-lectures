\section{Основы математического анализа}

Математический анализ - изучение через размышление

Объект математического анализа - \textit{функция}

В математическом анализе используются символы из математической логики и теории множеств.

\subsection{Математическая логика}

Объект изучения математической логики - высказывание.

\begin{definition}
  Высказывание - повествовательное предложение, относительно которого можно сказать, истинно оно или ложно. Обозначаются заглавными буквами латинского алфавита.
\end{definition}
\begin{eg}
  $2+3=5$ - истинно, $3 < 0$ - ложно
\end{eg}

\subsubsection{Логические символы}

\begin{itemize}
  \item $\land$ - конъюнкция (логическое "И")
  \item $\lor$ - дизъюнкция (логическое "ИЛИ")
  \item $\implies$ - импликация ("если А то В")
  \item $\iff$ - эквивалетность или равносильность ("тогда и только тогда")
\end{itemize}

Кванторы - общее название для логических операций

\begin{itemize}
  \item $\exists$ - существует
  \item $\nexists$ - не существует
  \item $!\exists$ - существует единствуенный элемент
  \item $\forall$ - для каждого
\end{itemize}

\subsection{Теория множеств}

\begin{definition}
  Множество - совокупность объектов, связанных одним и тем же свойством. Обозначаются заглавными латинскими буквами. Элементы множества обозначаются строчными латинскими буквами.
\end{definition}

\subsubsection{Символы теории множеств}

\begin{itemize}
  \item $\in$ - принадлежит
  \item $\notin$ - не принадлежит
  \item $\subset$ - включает
  \item $\subseteq$ - включает, возможно равенство
  \item $\equiv$ - тожденственное равенство (для любого значения переменной)
  \item $\emptyset$ - пустое множество
\end{itemize}

\subsubsection{Операции со множествами}

\begin{itemize}
  \item $\cup$ - объединение множеств
  \item $\cap$ - пересечение множеств
\end{itemize}

\begin{note}
   \[
  A\cup B = \{ x : x\in A \land x\in B\} 
  A\cap B = \{x : x\in A \lor x\in B\} 
  .\] 
\end{note}

\begin{definition}
  Подмножество - множество $A$ называется подмножеством  $B$, если каждый элемент множества $A$ является элементом множества $B$.
\end{definition}

\begin{definition}
  Универсальное множество - такое множество, подномножествами которого являются все рассматриваемые множества.
\end{definition}

\subsubsection{Способы задания множества}

\begin{enumerate}
  \item Перечислить все элементы:
   \[
  A = \{1, 2, 3, 4 \ldots\} 
  .\] 
  \item Указание свойства, которым обладают все элементы множества:
  \[
  B = \{x : Q(x)\} 
  .\] 
\end{enumerate}

\subsubsection{Числовые множества}
\begin{itemize}
  \item $\N = \{1, 2, 3, 4\} $ - множество натуральных чисел
  \item $\Z = \{ \ldots -2, -1, 0, 1, 2, \ldots$ \} - множество целых чисел
  \item $\Q = \{x : x = \frac{m}{n}, m\in \Z n\in \N\} $ - множество рациональных чисел
  \item $I = \{\pi, \sqrt{2}\ldots \} $ - множество иррациональных чисел
  \item $\R = \Q \cup I $ - множество действительных чисел
\end{itemize}
\begin{note}
  Порядок вложенности:
  \[
  \N \subset \Z \subset \Q \subset \R
  .\] 
\end{note}

\subsubsection*{Промежутки}
\begin{definition}
  Промежуток - подножество $X$ множества $\Q$, где \\ $\forall x_1, x_2\in X$ этому множеству принадлежат все $x$, где $x_1 < x < x_2$.
\end{definition}

\subsubsection{Виды промежутков}
\begin{enumerate}
  \item Отрезок $[a;b] = \{x\in \R : a \le x \le b\} $
  \item Интервал $(a; b) = \{x\in \R : a < x < b\} $
  \item Полуинтервал $[a; b) = \{x\in \R: a < x < b\} $
\end{enumerate}

\subsubsection{Конечные и бесконечные окрестности}

Пусть $x_0\in \R$, $\delta$ и $\epsilon$ - малые положительные величины

\begin{definition}
  Окрестностью точки $x_0$ называется любой интервал, содержаший эту точку
\end{definition}

\begin{definition}
  $\delta$ - окрестностью ($S(x_0, \delta$) точки $x_0$ называется интервал с центром в точке $x_0$ и длиной 2$\delta$.
  \[
  S(x_0; \delta) = (x_0 - \delta; x_0 + \delta)
  \] 
\end{definition}

\begin{definition}
  $\varepsilon$ - окрестностью ($S(x_0, \varepsilon$) точки $x_0$ называется интервал с центром в точке $x_0$ и длиной 2$\varepsilon$.
  \[
  S(x_0; \varepsilon) = (x_0 - \varepsilon; x_0 + \varepsilon)
  \] 
\end{definition}

\begin{definition}
  Окрестностью $+\infty$ называется любой интервал вида:
  \[
  S(+\infty) = (a; +\infty), a\in \R, a > 0
  .\] 
\end{definition}

\begin{definition}
  Окрестностью $-\infty$ называется любой интеграл вида:
  \[
  S(-\infty) = (-\infty; -a), a\in \R, a > 0
  .\] 
\end{definition}

\begin{definition}
  Окрестностью $\infty$ называется любой интервал вида
  \[
  S(\infty) = (-\infty; -a) \cup (a; +\infty), a\in \R, a > 0
  .\] 
\end{definition}

