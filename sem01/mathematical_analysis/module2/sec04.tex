\section{Правило Лопиталя-Бернулли}

\begin{theorem}
  Пусть $f(x)$ и  $\varphi(x)$ удовлетворяют условиям:
  \begin{itemize}
    \item Определены и дифференцируемы в $\mathring{S}(x_0)$
    \item $\lim_{x \to x_0} f(x) = 0, \lim_{x \to x_0} \varphi(x) = 0$
    \item $\forall x \in \mathring{S}(x_0) \quad \varphi'(x) \neq 0$
    \item $\exists \lim_{x \to x_0} \frac{f'(x)}{\varphi'(x)} = A$
  \end{itemize}
  Тогда $\exists \lim_{x \to x_0} \frac{f(x)}{\varphi(x)} = \lim_{x \to x_0} \frac{f'(x)}{\varphi'(x)} = A$.
\end{theorem}
\begin{proof}
  Доопределим функции $f(x)$ и $\varphi(x)$ в точке $x_0$ нулём: \[
  f(x_0) = 0 \quad \varphi(x_0) = 0
  \] 
  По условию:
  \begin{align*}
    &\lim_{x \to x_0} f(x) = 0 = f(x_0)
    &\lim_{x \to x_0} \varphi(x) = 0 = \varphi(x_0)
  \end{align*}
  
  $f(x)$ и  $\varphi(x)$ непрерывны в точке $x_0$.\\
  По условию функция $f(x)$ и  $\varphi(x)$ дифференцируемы в точке $\mathring{s}(x_0)$ $\implies$ по теореме о связи дифференцируемости и непрерывности $\implies f(x)$ и $\varphi(x)$ непрерывны в $\mathring{s}(x_0)$. Таким образом $f(x)$ и  $\varphi(x)$ непрерывны в $S(x_0)$.

  Функции $f(x)$ и  $\varphi(x)$ удовлетворяют условию т.Коши на $[x_0, x]$. Тогда по теореме Коши $\implies$ 
  \begin{gather*}
    \exists c \in [x_0, x] : \frac{f(x) - f(x_0)}{\varphi(x) - \varphi(x_0)} = \frac{f'(c)}{\varphi'(c)} \tag{*} 
  \end{gather*}

  Т.к. $f(x_0) = 0$ и $\varphi(x_0) = 0$ $\implies$ \[
    (*) \quad \boxed{\frac{f(x)}{\varphi(x)} = \frac{f'(c)}{\varphi(c)}}
  \] 

  Т.к. $\exists \lim_{x \to x_0} \frac{f'(x)}{\varphi'(x)} = A \implies$ правая часть (*): \[
  \lim_{c \to x_0} \frac{f'(c)}{\varphi'(c)} = A
  \] 
  Левая часть (*): \[
  \lim_{x \to x_0} \frac{f(x)}{\varphi(x)} = \lim_{x \to x_0} \frac{f'(c)}{\varphi'(c)} = A
  \] 

  Получаем: \[
  \lim_{x \to x_0} \frac{f(x)}{\varphi(x)} = \lim_{x \to x_0} \frac{f'(x)}{\varphi'(x)} = A
  \] 
\end{proof}

\begin{theorem}
  Пусть $f(x)$ и  $\varphi(x)$ удовлетворяют условиям:
  \begin{itemize}
    \item Определены и дифференцируемы в $\mathring{S})(x_0)$
    \item $\lim_{x \to x_0} f(x) = \infty, \lim_{x \to x_0} \varphi(x) = \infty$
    \item $\forall x \in \mathring{S}(x_0) \quad \varphi'(x) \neq 0$
    \item $\exists \lim_{x \to x_0} \frac{f'(x)}{\varphi'(x)} = A$
  \end{itemize}
  Тогда $\exists \lim_{x \to x_0} \frac{f(x)}{\varphi(x)} = \lim_{x \to x_0} \frac{f'(x)}{\varphi'(x)} = A$.
\end{theorem}

\subsection{Сравнение показательной, степенной и логарифмической функции на бесконечности}

Пусть:
\begin{align*}
  f(x) &= x^n \\
  g(x) &= a^x \\
  h(x) &= \ln x \\
\end{align*}

Найдём предел при стремлении к бесконечности:
\begin{align*}
  \lim_{x \to +\infty} \frac{f(x)}{g(x)} &= \lim_{x \to +\infty} \frac{x^n}{a^x} = \left( \frac{\infty}{\infty} \right) = \lim_{x \to +\infty} \frac{n \cdot x ^ {n-1}}{a^x \ln a} \\
                                  &= \left( \frac{\infty}{\infty} \right) = \ldots = \lim_{x \to +\infty} \frac{n(n-1)(n-2)\ldots \cdot 1}{a^x(\ln a)^n} = \\
                                  &= \frac{n!}{\ln^n a} \lim_{x \to +\infty} \frac{1}{a^x} = \frac{n!}{\ln^n a} = 0.
\end{align*}

Значит $a^x$ растёт быстрее, чем  $x^n$ при $x \to \infty$ или $x^n = o(a^x)$ при $x \to +\infty$.

Найдём предел при стремлении к бесконечности:
\begin{align*}
  \lim_{x \to +\infty} \frac{h(x)}{f(x)} = \lim_{x \to +\infty} \frac{\ln x}{x^n} = \left( \frac{\infty}{\infty} \right) \\
  &= \lim_{x \to +\infty}  \frac{\frac{1}{x}}{n \cdot x^{n-1}} = \frac{1}{n} \lim_{x \to +\infty} \frac{1}{x^n} = \frac{1}{n} \cdot 0 = 0
\end{align*}

Значит, $x^n$ растёт быстрее, чем  $\ln x$ при $x\to +\infty$ $\ln x = o(x^n)$ при $x \to  +\infty$.

Вывод: на бесконечности функции расположены в таком порядке:
\begin{enumerate}
  \item $g(x) = a^x$ -- самая быстрорастущая функция \\
  \item $f(x) = x^n$ \\
  \item $h(x) = \ln x$
\end{enumerate}

