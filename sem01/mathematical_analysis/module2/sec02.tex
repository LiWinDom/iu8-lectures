\subsection{Уравнение касательной и нормали к графику функции.}

Пусть $f(x)$ опредена в $S(x_0)$. 
Обозначим:
\begin{itemize}
  \item $f(x_0) = y_0$, $M(x_0, y_0)$
  \item $\Delta x$ -- приращение функции
  \item $x = x_0 + \Delta x$ 
  \item $N(x0 + \Delta x, y(x_0 + \Delta x))$
  \item MN - секущая
\end{itemize}

При $\Delta x \to 0$ точка N движется вдоль графика функции $y = f(x)$, а секущая MN вращается вдоль графика.

В пределе $\lim_{\Delta x \to 0}$ секущая MN становиться \textit{касательной}.

\begin{definition}
  Если существует предельное секущей MN, когда точка N перемещается вдоль графика функции к точке M, это положение называется \textit{касательной} к графику функции в точке $M$.
\end{definition}

\begin{gather*}
  \triangle MNK: \tg \alpha = \frac{\Delta y}{\Delta x} \\
  \begin{rcases}
    \lim_{\Delta x \to 0} \tg \alpha_0 \\
    \lim_{\Delta x \to 0} y'(x_0)
  \end{rcases} \implies \boxed{
    \tg \alpha_0 = y'(x_0)
  }
\end{gather*}

где $\alpha$ -- угол между секущей и положительным направлением оси OX, а
$\alpha_0$ -- угол между касательной и положительным направлением оси OX.

С другой стороны, прямая, проходящая через точку $M_0(x_0, y_0)$ с заданным угловым коэффициентом $k$ имеет вид: \[
y - y_0 = k(x - x_0)
\] 
где $k$ -- тангенс угла наклона прямой к положительному направлению оси Ox.  \[
\tg \alpha_0 = y'(x_0) = k
\] 

Рассмотрим $\forall P(x, y)$ на касательной к графику функции $y = f(x)$ в точке $M(x_0, y_0)$:
\begin{gather*}
  \triangle MPK: \tg \alpha = \frac{PK}{MK} \\
  \tg \alpha_0 = \frac{y - y_0}{x - x_0} \\
  y'(x_0) = \tg \alpha_0 \\
  y'(x_0) = \frac{y - y_0}{x - x_0}
\end{gather*}

Получаем: \[
  \boxed{y - y_0 = y'(x_0)(x - x_0)}
\] 
-- уравнение касательной к графику функции $y = f(x)$ в точке $M(x_0, y_0)$

Выводы:
\begin{enumerate}
  \item Геометрический смысл производной: производная функции $y = f(x)$ в точке  $x_0$ равна тангенсу угла наклона касательной к положительному направлению оси Ox или угловому коэффициенту касательной. \[
    y'(x_0) = \tg \alpha_0 = k
  \] 
  \item Механический смысл производной функции $s = f(t)$ в точке $t_0$ равна мгновенной скорости в момент $t_0$ \[
      V(t_0) = s'(t_0) 
    \]
\end{enumerate}

\begin{definition}
  \textit{Нормалью} к графику функции $y=f(x)$ называется прямая, перпендикулярная касательной к графику функции в данной точке.
\end{definition}

\begin{gather*}
  l_1: y_1 = k_1x + b_1 \\
  l_2: y_2 = k_2x + b_2 \\
  l_1 \perp l_2 \iff k_1 \cdot k_2 = -1
\end{gather*}

\begin{gather*}
    y - y_0 = y'(x)(x - x_0) \\
    k_1 = y'(x) \implies k_2 = -\frac{1}{y'(x)} \implies \\
    \boxed{
      y - y_0 = -\frac{1}{y'(x)}(x - x_0)
    }
\end{gather*}

\begin{note}
  Касательная к графику функции существует не в любой точке (точка излома и точка заострения).
\end{note}

\begin{definition}
  Кривая, имеющая касательную в любой точке рассматриваемого промежутка, называется \textit{гладкой}.
\end{definition}

\begin{corollary}
  Если $y'(x_0) = \infty$, то касательная к графику функции $y=f(x)$ в точке $x_0$, параллельно оси ординат и имеет вид $x = x_0$ (нормаль имеет вид $y = y_0$). \\
  Если $y'(x_0) = 0$, то касательная к графику функции $y = f(x)$ в точке  $x_0$ имеет вид $y = y_0$ (нормаль имеет вид $x = x_0$).
\end{corollary}

\begin{definition}
  \textit{Углом между двумя пересекающимися кривыми} в точке с абциссой $x_0$ называется угол между касательными, проведёнными в этой точке.
\end{definition}

\begin{corollary}
  \begin{align*}
    \begin{matrix}
      y &= f_1(x) \\
      y &= f_2(x) \\
    \end{matrix}
    \implies \quad f_1 \cap f_2 = M_0(x_0, y_0) \quad 
    \begin{matrix}
      y_1 &= k_1x + b_1 \\
      y_2 &= k_2x + b_2
    \end{matrix} \\
  \end{align*} 
  \begin{gather*}
    \varphi \text{ -- угол между } f_1, f_2
    \varphi = \alpha_2 - \alpha_1 \\
    \tg \alpha_1 = k_1 = f_1(x_0) \\
    \tg \alpha_2 = k_2 = f_2(x_0) \\
    \tg \varphi = \tg(\alpha_2 - \alpha_1) = \frac{\tg \alpha_2 - \tg \alpha_1}{1 + \tg \alpha_2 \cdot \tg \alpha_1}
    \frac{k_2 - k_1}{1 + k_2 k_1} = \frac{f'_2(x_0) f
    _1(x_0)}{1 + f
    _1(x_0) \cdot f'_2(x_0)} \\
    \boxed{
      \tg \varphi = |\frac{f'_2(x_0) - f'_1(x_0)}{1 + f'_2(x_0) f'_1(x_0)}| 
    }
  \end{gather*}
\end{corollary}

