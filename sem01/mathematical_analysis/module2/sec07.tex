\begin{theorem}
  \textit{Достаточное условие выпуклости функции}. \\
  Пусть функция $y = f(x)$ дважды дифференцируема на интервале  $(a, b)$. 
  Тогда: 
  \begin{enumerate}
    \item Если  $f''(x) < 0 \forall  x \in (a, b)$, то график функции \textit{выпуклый вверх} на этом интервале
    \item Если  $f''(x) > 0 \forall  x \in (a, b)$, то график функции \textit{выпуклый вниз} на этом интервале
  \end{enumerate}
\end{theorem}
\begin{proof}
  \begin{gather*}
    x_0 \in (a, b), y_0 = f(x_0) \implies M_0(x_0, y_0)
  \end{gather*}
  Построим в точке $M_0$ касательную к графику функции $y = f(x)$. Запишем уравнение касательной:
   \begin{gather*}
    y = y_0 = y'(x_0)(x - x_0) 
  \end{gather*}
  Преобразуем:
  \begin{gather*}
    y_k = f(x_0) + f'(x_0)(x - x_0) \tag{0} 
  \end{gather*}
  Представим функцию $y=f(x)$ по формуле Тейлора с остаточным членом в форме Лагранжа.
  \begin{gather*}
    f(x) = f(x_0) + \frac{f'(x_0)}{1!}x + \frac{f''(c)}{2!}(x - x_0)^2, \quad c \in S(x_0) \tag{2} 
  \end{gather*}
  Вычтем (1) из (2):
  \begin{gather*}
    f(x) - y_k = f(x_0) + \frac{f'(x_{0})}{1!}(x - x_0) + \frac{f''(c)}{2!}(x - x_0) - f(x_0) - f'(x_0)(x - x_0)^2 \\
    f(x) - y_k = \frac{f''(c)}{2!}(x - x_0)^2
  \end{gather*}
  1) По условию $f''(x) < 0 \forall x \in (a, b)$, то $f''(c) < 0 \implies f(x) - y_0 < 0 \implies f(x) < y_k$, а значит по определению выпуклой функции $\implies$ график функции $y = f(x)$ выпуклый вверх.
  2) По условию $f''(x) > 0 \forall x \in (a, b)$, то $f''(c) > 0 \implies f(x) - y_0 > 0 \implies f(x) > y_k$, а значит по определению выпуклой функции $\implies$ график функции $y = f(x)$ выпуклый вниз.
\end{proof}

\begin{theorem}
  \textit{Необходимое условие существование точки перегиба}. \\
  Пусть функция $y = f(x)$ в точке $x_0$ имеет \textit{непрерывную} вторую производную и $M(x_0, y_0)$ -- точка перегиба графика функции $y = f(x)$. Тогда $f''(x_0) = 0$.
\end{theorem}
\begin{proof}
  Докажем методом от противного. 
  Предположим, что $f''(x_0) > 0$. В силу непрерывности второй производной функции  $y = f(x)$ $\exists  S(x_0) \forall  x \in S(x_0) : f''(x) > 0$. Это противоречит тому, что $M_0(x_0, y_0)$ -- точка перегиба.
  Предположим, что $f''(x_0) < 0$. В силу непрерывности второй производной функции  $y = f(x)$ $\exists  S(x_0) \forall  x \in S(x_0) : f''(x) < 0$. Это противоречит тому, что $M_0(x_0, y_0)$ -- точка перегиба.
\end{proof}

\begin{definition}
  Точки из области определения функции, в которых вторая производная функции равна нулю или не существует, называются \textit{критическими точками} второго порядка.
\end{definition}

\begin{theorem}
  \textit{Достаточное условие существования точки перегиба}. \\
  Если функция $y = f(x)$ непрерывна в точке $x_0$, дважды дифференцируема в $S(x_0)$ и вторая производная меняет знак при переходе аргумента $x$ через точку $x_0$. Тогда $M_0(x_0, f(x_0))$ является точкой перегиба графика функции $y = f(x)$.
\end{theorem}
\begin{proof}
  По условию $\exists S(x_0)$ в которой вторая производная функции $y = f(x)$ меняет знак при переходе аргумента $x$ через точку $x_0$ (даёт достаточное условие выпуклости функции).
  Это означает, что график функции $y = f(x)$ имеет различные направление выпуклости по разные стороны от точки $x_0$.
  По определению точки перегиба $M(x_0, f(x_0))$ является точкой перегиба графика функции $y = f(x)$.
\end{proof}

