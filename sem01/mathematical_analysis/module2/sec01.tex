\section{Дифференциальное исчисление функции одной переменной}

Рассмотрим $y=f(x)$ определённую в $S(x_0)$. Пусть $x$ -- произвольная точка из $S(x_0)$.
Обозначим:
\begin{itemize}
  \item $\Delta x$ -- приращение аргумента \[
    x = x_0 + \Delta x \quad \implies \quad \Delta x = x - x_0
  \]
  \item $\Delta y$ -- приращение функции \[
    \Delta y = y(x_0 + \Delta x) - y(x_0)
  \] 
\end{itemize}

\begin{definition}
  \textit{Производной функции $y = f(x)$ в точке $x_0$ } называется предел отношения приращения функции и предел приращения аргумента при стремлении последнего к нулю. \[
  y'(x_0) = \lim_{\Delta x \to 0} \frac{\Delta y}{\Delta x}
  \] 
\end{definition}

Если предел \textit{конечен}, то функция $y=f(x)$ в точке $x_0$ имеет конечную производную.
Если предел \textit{бесконечен}, то функция $y=f(x)$ в точке $x_0$ имеет бесконечную производную.

\textit{Дифференцирование} -- процесс получения производной.

\begin{eg}
  \begin{gather*}
    y = e^x, D_f = \R \\
    x = x_0 + \Delta x, \forall x \in D_f \\
\Delta  y = y(x_0 + \Delta x) - y(x_0) = e^{x_0 + \Delta x} - e^{x_0} = e^{x_0} \left(e^{\Delta x} - 1\right) \\
y'(x_0) = \lim_{\Delta x \to 0} \frac{\Delta y}{\Delta x} = \lim_{\Delta x \to 0} \frac{e^{x_0} \left( e^{\Delta x} - 1 \right) }{\Delta x} = \lim_{\Delta x \to 0} \frac{e^{x_0} \Delta x}{\Delta x} = e^{x0}
  \end{gather*}
\end{eg}

\pagebreak

\begin{eg}
  \begin{gather*}
    y = \sin(x), D_f = \R \\
    x = x_0 + \Delta x, \forall x \in D_f \\
  \Delta y = y(x_0 + \Delta x) + y(x_0) = \sin(x_0 + \Delta x) - \sin(x_0) = \\
  2\sin \frac{x_0 + \Delta x - x_0}{2} \cos \frac{x_0 + \Delta x + x_0}{2} = 2\sin \frac{\Delta x}{2} \cos \left(x_0 + \frac{\Delta x}{2}\right) \\
  y'(x_0) = \lim_{\Delta x \to 0} \frac{\Delta y}{\Delta x} = \lim_{\Delta x \to 0} \frac{2\sin \frac{\Delta x}{2} \cos \left( x_0 + \frac{\Delta x}{2} \right) }{\Delta x} = \\
  \lim_{\Delta x \to 0} \frac{2 \cdot \frac{\Delta x}{2} \cos\left( x_0 + \frac{\Delta x}{2} \right) }{\Delta x} = \lim_{\Delta x \to 0} \cos\left( x_0 + \frac{\Delta x}{2} \right) = \cos x_0
  \end{gather*}
\end{eg}

\subsection{Односторонние производные}

\begin{definition}
  \textit{Производной функции $y=f(x)$ в точке $x_0$ справа} или \textit{правосторонней производной} называется предел отношения приращения функции к приращению аргумента при стремлении к нулю справа. \[
  y'_+(x_0) = \lim_{\Delta x \to 0+} \frac{\Delta y}{\Delta x}
  \] 
\end{definition}

\begin{definition}
  \textit{Производной функции $y=f(x)$ в точке $x_0$ слева} или \textit{левосторонней производной} называется предел отношения приращения функции к приращению аргумента при стремлении к нулю слева. \[
  y'_-(x_0) = \lim_{\Delta x \to 0-} \frac{\Delta y}{\Delta x}
  \] 
\end{definition}

\begin{theorem}
  \textit{О существовании производной функции в точке}.
  Функция $y = f(x)$ в точке $x_0$ имеет производную тогда и только тогда, коогда она имеет производные и справа, и слева, и они равны между собой. \[
  y'(x_0) = y'_+(x_0) = y'_-(x_0)
  \] 
\end{theorem}

\begin{eg}
  \begin{gather*}
    y = |x|, x_0 = 0 \\
    y = \begin{cases}
      x, x > 0 \\
      0, x = 0 \\
      -x, x < 0
    \end{cases}
    \implies y' = \begin{cases}
      1, x > 0 \\
      0, x = 0 \\
      -1, x < 0
    \end{cases} \\
    \begin{rcases}
      y'_+(0) = 1 \\
      y'_-(0) = -1
    \end{rcases} \text{ -- т.к. производные конечные, но различные, то } x_0 = 0 \text{ называется точкой излома}
  \end{gather*}
  \textbf{Геометрический смысл: } $\not \exists $ касательной к функции в точке излома. 
\end{eg}

\begin{eg}
  \begin{gather*}
    y = x^{\frac{1}{3}}, z_0 = 0 \\
    y' = \frac{1}{3}x{-\frac{2}{3}}
  \end{gather*}   
\end{eg}
