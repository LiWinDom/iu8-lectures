\section{Сравнение бесконечно малых и бесконечно больших функций}

Пусть даны функции $\alpha(x)$ и $\beta(x)$, которые являются б.м.ф. при $x \to x_0$. \[
  \lim_{x \to x_0} \frac{\alpha(x)}{\beta(x)}
\]

Рассмотрим варианты:

\begin{itemize}

  \item \[
      \lim_{x \to x_0} \frac{\alpha(x)}{\beta(x)} = 0
  \] 
  \begin{center}
    $\alpha(x)$ имеет более высокий порядок малости, чем $\beta(x)$. \\
    $\boxed{\alpha(x) = o(\beta(x)) \text{, при } x \to x_0}$
  \end{center} 
  
  \item \[
  \lim_{x \to x_0} \frac{\alpha(x)}{\beta(x)} = \infty
  \]
  \begin{center}
    $\beta(x)$ имеет более высокий порядок малости, чем $\alpha(x)$. \\
    \boxed{\beta(x) = o(\alpha(x)) \text{, при } x \to x_0}
  \end{center} 

  \item \[
    \lim_{x \to x_0} \frac{\alpha(x)}{\beta(x)} = 1
  \] 
  \begin{center}
    $\alpha(x)$ и $\beta(x)$ - эквивалентны. \\
    \boxed{\alpha(x) \sim \beta(x) \text{, при } x \to  x_0}
  \end{center}

  \item \[
    \lim_{x \to x_0} \frac{\alpha(x)}{\beta(x)} = const
  \] 
  \begin{center}
    $\alpha(x)$ и $\beta(x)$ - одного порядка малости.
  \end{center}
  \begin{gather*}
    \boxed{
      \begin{matrix}
        \alpha(x) = O(\beta(x)) \\
        \beta(x) = O(\alpha(x))
      \end{matrix} \quad
      \text{ при } x \to x_0
    }
  \end{gather*}
  \\
  \item \[
  \not\exists \lim_{x \to x_0} \frac{\alpha(x)}{\beta(x)}
  \] 
  \begin{center}
    $\alpha(x)$ и $\beta(x)$ - несравнимы.
  \end{center}
\end{itemize}

\begin{definition}
  Две б.м.ф. $\alpha(x)$ и $\beta(x)$ называются одного порядка малости, если: \[
  \lim_{x \to x_0} \frac{\alpha(x)}{\beta(x)} = const \neq 0
  \] 
\end{definition}

\begin{definition}
  Две б.м.ф. $\alpha(x)$ и $\beta(x)$ называются \textit{несравнимыми} , если: \[
  \not \exists \lim_{x \to x_0} \frac{\alpha(x)}{\beta(x)}
  \] 
\end{definition}

\begin{definition}
  Две б.м.ф. $\alpha(x)$ и $\beta(x)$ называются \textit{эквивалентными} , если: \[
  \lim_{x \to x_0} \frac{\alpha(x)}{\beta(x)} = 1
  \] 
\end{definition}


\begin{definition}
  Если: \[
  \lim_{x \to x_0} \frac{\alpha(x)}{\beta(x)} = 0
  \]
  где $\alpha(x)$ и $\beta(x)$ -- б.м.ф. при $x \to x_0$, то говорят, что функция $\alpha(x)$ имеет более высокий порядок малости, чем $\beta(x)$.
\end{definition}

\begin{definition}
  Б.м.ф. $\alpha(x)$ имеет порядок малости $k$ относительно функции б.м.ф.  $\beta(x)$, если: \[
    \lim_{x \to x_0} \frac{\alpha(x)}{[\beta(x)]^k} = const \neq 0
  \]
  где $k$ -- порядок малости.
\end{definition}

\subsection{Свойства эквивалентных бесконечно малых функций}

\begin{theorem}
  Если $\alpha(x) \sim \beta(x)$, а $\beta(x) \sim \gamma(x)$, при $x \to x_0$, то $\alpha(x) \sim \gamma(x)$ при $x \to x_0$.
\end{theorem}
\begin{proof}
  \begin{gather*}
    \lim_{x \to x_0} \frac{\alpha(x)}{\gamma(x)} = \lim_{x \to x_0} \frac{\alpha(x) \cdot \beta(x)}{\gamma(x) \cdot \beta(x)}
    = \lim_{x \to 0} \frac{\alpha(x)}{\beta(x)} \cdot \frac{\beta(x)}{\gamma(x)}
    = 1 \cdot 1 = 1 \\
    \implies \alpha(x) \sim \gamma(x) \text{, при } x \to x_0
  \end{gather*}
\end{proof}

\begin{theorem}
  \textit{Необходимое и достаточное условие экваивалентных \\ бесконечно малых функий.} \\
  Две функции $\alpha(x)$ и $\beta(x)$ эквивалентны тогда и только тогда, когда их разность имеет более высокий порядок малости по сравнению с каждой из них.
  \begin{gather*}
    \alpha(x), \beta(x) \text{ - б.м.ф при } x \to x_0 \\
    \alpha(x) \sim \beta(x) \iff 
    \begin{matrix}
    \alpha(x) - \beta(x) = o(\alpha(x)) \\
    \alpha(x) - \beta(x) = o(\beta(x))
    \end{matrix}
    \quad \text{при } x \to x_0
  \end{gather*}
\end{theorem}
\begin{proof}
  Необходимость. \\
  Дано: \[
  \alpha(x), \beta(x) \text{ - б.м.ф при } x \to x_0
  \] 
  Доказать: \[
  \alpha(x) - \beta(x) = o(\alpha(x)) \text{, при } x \to x_0
  \] 
  Доказательство:
  \begin{align*}
    \lim_{x \to x_0} \frac{\alpha(x) - \beta(x)}{\alpha(x)} &= \lim_{x \to x_0} \left( 1 - \frac{\beta(x)}{\alpha(x)} \right) \\
    &= 1 - \lim_{x \to x_0} \frac{\beta(x)}{\alpha(x)} \\
    &= 1 - \frac{1}{1} = 0
  \end{align*}

  Достаточность. \\
  Дано: \[
  \alpha(x) - \beta(x) = o(\beta(x)) \text{, при } x \to x_0
  \]
  Доказать: \[
  \alpha(x) \sim \beta(x) \text{, при } x \to x_0
  \] 
  Доказательство:
  \begin{align*}
    \lim_{x \to x_0} \frac{\alpha(x) - \beta(x)}{\beta(x)} &= \lim_{x \to x_0} \left( \frac{\alpha(x)}{\beta(x)} - 1 \right)  \\
    &= \lim_{x \to x_0} \frac{\alpha(x)}{\beta(x)} - 1 = 0
  \end{align*}
  \begin{gather*}
    \implies \lim_{x \to x_0} \frac{\alpha(x)}{\beta(x)} = 1 \\
    \implies \alpha(x) \sim \beta(x) \text{, при } x \to x_0 
  \end{gather*}
\end{proof}

\begin{theorem}
  \textit{О суммы бесконечно малых разного порядка}. \\
  Сумма бесконечно малых функций разных порядком малости эквивалентно слагаемому низшего порядка малости.
  \begin{gather*}
    \begin{rcases}
      \alpha(x), \beta(x) \text{ - б.м.ф при } x \to x_0 \\
      \alpha(x) = o(\beta(x)) \text{, при } x \to x_0
    \end{rcases} 
    \implies \alpha(x) + \beta(x) \sim \beta(x) \text{, при } x \to x_0
  \end{gather*}
\end{theorem}
\begin{proof}
  Рассмотрим предел: 
  \begin{align*}
    \lim_{x \to x_0} \frac{\alpha(x) + \beta(x)}{\beta(x)} &= \lim_{x \to x_0} \left( \frac{\alpha(x)}{\beta(x)} + 1 \right)  \\
    &= \lim_{x \to x_0} \frac{\alpha(x)}{\beta(x)} + 1 \\
    &= 0 + 1 = 1
  \end{align*}
\end{proof}

\begin{corollary}
  Сумма б.б.ф. разного порядка роста эквивалентна слагаемому высшего порядка роста. 
\end{corollary}

\begin{theorem}
  \textit{О замене функции на эквивалентную под знаком предела}. \\ 
  Предел \textbf{отношения} двух б.м.ф. (б.б.ф) не изменится, если заменить эти функции на эквивалентные. \[
  \begin{rcases}
    \alpha(x), \beta(x) \text{ - б.м.ф. при } x \to x_0 \\
    \alpha(x) \sim \alpha_0(x) \\
    \beta(x) \sim \beta_0(x)
  \end{rcases} \implies 
  \lim_{x \to x_0} \frac{\alpha(x)}{\beta(x)} = \frac{\alpha_0(x)}{\beta(x)} 
  \] 
\end{theorem}
\begin{proof}
  Рассмотрим предел:
  \begin{align*}
    \lim_{x \to x_0} \frac{\alpha(x)}{\beta(x)} &= \lim_{x \to x_0} \frac{\alpha(x) \cdot \alpha_0(x) \cdot \beta_0(x)}{\beta(x) \cdot \alpha_0(x) \cdot \beta_0(x)} \\
    &= \lim_{x \to x_0} \frac{\alpha(x)}{\alpha_0(x)} \cdot \lim_{x \to x_0} \frac{\beta_0(x)}{\beta(x)} \cdot \lim_{x \to x_0} \frac{\alpha_0(x)}{\beta_0(x)} \\
    &= 1 \cdot  1 \cdot 1 \cdot \lim_{x \to x_0} \frac{\alpha(x)}{\beta(x)} \\
  \end{align*}
\end{proof}

\begin{table}[htpb]
  \centering
  \caption{Таблица эквивалентных б.м.ф}
  \label{tab:label}
  \begin{tabular}{ c  c }
    1-ый замечательный предел & 2-ой замечательный предел \\
    \\
    $\sin(x) \sim x$ при $x \to  0$ 
                              & \\
    $\tg(x) \sim x$ при $x \to 0$
                              & $\ln(1+x) \sim x$ \\
    $\arcsin(x) \sim x$ при $x \to  0$
                              & $\log_a(1 + x) \sim \frac{x}{\ln a}$ \\
    $\arctg(x) \sim x$ при $x \to  0$
                              & $e^x$ ~ x \\
    $1 - \cos(x) \sim \frac{x^2}{2}$ при $x \to 0$
                              & $a^x - 1 \sim x \ln a$ \\
    $1 - \cos(x) \sim \frac{x^2}{2}$ при $x \to  0$
                              & \\
                              \\
    \multicolumn{2}{ c }{
      Сумма б.м.ф. и б.б.ф.
    } \\
    \\
    \multicolumn{2}{ c }{
      $a_0 + a_1x + a_2x^2 + \ldots + a_n x^n \sim a_n x^n$ при $x \to \infty$
    } \\
    \multicolumn{2}{ c }{
      $a_1x + a_2x^2 + \ldots + a_n x^n \sim a_1x$ при $x \to 0$
    } \\
  \end{tabular}
\end{table}

