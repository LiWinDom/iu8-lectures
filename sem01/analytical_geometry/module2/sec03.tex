\section{Системы линейных алгебраических уравнений (СЛАУ)}

\begin{definition}
  \textit{Системой линейных алгебраических уравнений} называется система уравнений вида:
  \begin{gather*}
    \begin{cases}
      a_{11} x_1  + a_{12} x_2 + \ldots a_{1n} = b_1 \\
      a_{21} x_1  + a_{22} x_2 + \ldots a_{2n} = b_2 \\
      \ldots \\
      a_{m1} x_1  + a_{m2} x_2 + \ldots a_{mn} = b_m \\
    \end{cases}
  \end{gather*}
  где $a_{ij} = const$, $i=1..m$, $j=1..n$ -- коэффициенты СЛАУ, $b_i = const$ -- свободный член СЛАУ,  $x_{i}, i=1..n$ -- неизвестная переменная СЛАУ.
\end{definition}

\begin{definition}
  Совокупность переменных $(x_1, x_2 \ldots x_{n})$, при которых каждое уравнение обращается в верное равенство, называется \textit{решением} данной СЛАУ. 
\end{definition}

Форма записи СЛАУ выше называется \textit{координатной}. 

\subsubsection*{Матричная форма}

Обозначим:
\begin{gather*}
  A = \left( 
\begin{matrix}
  a_{11} & a_{12} & \ldots & a_{1n} \\
  a_{21} & a_{22} & \ldots & a_{2n} \\
  \ldots & \ldots & \ldots & \ldots \\
  a_{m1} & a_{m2} & \ldots & a_{mn} \\
\end{matrix}
  \right) \\
  B = \left( 
    \begin{matrix}
      b_1 \\
      b_2 \\
      \ldots \\
      b_m
    \end{matrix}
  \right) \qquad
  X = \left( 
  \begin{matrix}
    x_1 \\
    x_2 \\
    \ldots \\
    x_n
  \end{matrix}
  \right) 
\end{gather*}

Тогда СЛАУ можно записать в виде: \[
  \boxed{A \times X = B}
\] 

\subsubsection*{Векторная форма записи}

Обозначим:
\begin{gather*}
  \vec{a}_1 = \left( 
    \begin{matrix}
      a_{11} \\
      a_{21} \\
      \ldots \\
      a_{m1} \\
    \end{matrix}
  \right) \qquad
  \vec{a}_2 = \left( 
    \begin{matrix}
      a_{21} \\
      a_{22} \\
      \ldots \\
      a_{2m}
    \end{matrix}
  \right) \qquad \ldots \qquad 
  \vec{a}_n = \left( 
    \begin{matrix}
      a_{n1} \\
      a_{n2} \\
      \ldots \\
      a_{nm}
    \end{matrix}
  \right) \qquad 
  \vec{b} = \left( 
    \begin{matrix}
      b_1 \\ 
      b_2 \\
      \ldots \\
      b_m
    \end{matrix}
  \right)
\end{gather*}

Тогда вектор $\vec{b}$, координаты которого являются свободные члены, можно представить в виде линейной комбинаций векторов $\vec{a}$, координаты которых соответствуют элементам столбцов матрицы.

\begin{gather*}
  \vec{a_1} x_1 + \vec{a_2} x_2 + \ldots + \ldots \vec{a_n} x_{n} = \vec{b}
\end{gather*}

\begin{definition}
  СЛАУ, имеющая решение, назыается \textit{совместной}.
\end{definition}

\begin{definition}
  СЛАУ, не имеющая решение, называется \textit{несовместной}.  
\end{definition}

\begin{definition}
  Совместная СЛАУ, имеющая единственное решение, называется \textit{совместно-определённой}. 
\end{definition}

\begin{definition}
  Совместная СЛАУ, имеющая бесконечное кол-во решений, называется \textit{совместно-неопределённой}.
\end{definition}

\begin{definition}
  СЛАУ, у которой все свободные члены равны нулю, называется \textit{однородной}.
\end{definition}

\begin{definition}
  СЛАУ, у которой хотя бы один свободный член не равен нулю, называется \textit{неоднородной}.
\end{definition}

\subsection{Решение матричных уравнений}

\subsection*{I}

Для уравнения вида: \[
A \times X = B
\] 
\begin{enumerate}
  \item Умножим обе части уравнения на обратную матрицу $A$ \textbf{слева}:
    \begin{align*}
      A^{-1} \times A \times X &= A^{-1} \times B \\
      E \times X &= A^{-1} \times B \\
      X &= A^{-1} \times  B
    \end{align*}
\end{enumerate}

\subsection*{II}

Для уравнений вида: \[
X \times A = B
\] 
\begin{enumerate}
  \item Умножим обе части уравнения на обратную матрицу $A$ \textbf{справа}:
    \begin{align*}
      X \times A \times A^{-1} &= B \times A^{-1} \\
      X \times E &= B \times A^{-1} \\
      X &= B \times A^{-1} \\
    \end{align*}
\end{enumerate}

\subsection*{III}

Для уравнения вида: \[
A \times X \times C = B
\] 
\begin{enumerate}
  \item Умножим обе части уравнения на обратную матрицу $A$ \textbf{слева} и на обратную матрицу $C$ \textbf{справа}:
  \begin{align*}
    A^{-1} \times  A \times  X \times C \times C^{-1} &= A^{-1} \times B \times C^{-1} \\
    E \times  X \times E &= A^{-1} \times B \times C^{-1} \\
    X &= A^{-1} \times B \times C^{-1}
  \end{align*}
\end{enumerate}

\subsection{Формулы Крамера для решения СЛАУ}

Запишем СЛАУ в матричном виде: 
\begin{gather*}
  A \times X = B \qquad A_{n \times n} \\
  A = \left( 
  \begin{matrix}
    a_{11} & a_{12} & \ldots & a_{1n} \\
    a_{21} & a_{22} & \ldots & a_{2n} \\
    \ldots & \ldots & \ldots & \ldots \\
    a_{n1} & a_{n2} & \ldots & a_{nn}
  \end{matrix}
  \right) 
\end{gather*}

Пусть матрица не вырожденная. Тогда её обратная матрица будет иметь вид: 
\begin{gather*}
  A^{-1} = \frac{1}{det A} \left( 
    \begin{matrix}
    A_{11} & A_{12} & \ldots & A_{1n} \\
    A_{21} & A_{22} & \ldots & A_{2n} \\
    \ldots & \ldots & \ldots & \ldots \\
    A_{n1} & A_{n2} & \ldots & A_{nn}
    \end{matrix}
  \right)^{\tau} = \left( 
  \begin{matrix}
    \frac{A_{11}}{det A} & \frac{A_{12}}{det A} & \ldots & \frac{A_{1n}}{det A} \\
    \frac{A_{21}}{det A} & \frac{A_{22}}{det A} & \ldots & \frac{A_{2n}}{det A} \\
    \ldots & \ldots & \ldots & \ldots \\
    \frac{A_{n1}}{det A} & \frac{A_{n2}}{det A} & \ldots & \frac{A_{nn}}{det A}
  \end{matrix}
  \right) \\
  \\
  X = \left( 
  \begin{matrix}
    \frac{A_{11}}{det A} & \frac{A_{12}}{det A} & \ldots & \frac{A_{1n}}{det A} \\
    \frac{A_{21}}{det A} & \frac{A_{22}}{det A} & \ldots & \frac{A_{2n}}{det A} \\
    \ldots & \ldots & \ldots & \ldots \\
    \frac{A_{n1}}{det A} & \frac{A_{n2}}{det A} & \ldots & \frac{A_{nn}}{det A}
  \end{matrix}\right) \cdot \left( 
  \begin{matrix}
    b_1 \\
    b_2 \\
    \ldots \\
    b_{n}
  \end{matrix}
\right) \\
  \\
  x_i = 
  \frac{A_{i1}}{det A} b_1 +
  \frac{A_{i2}}{det A} b_2 +
  \ldots +
  \frac{A_{in}}{det A} b_n = \\
  = \frac{A_{i1}b_1 + A_{i_2}b_2 + \ldots + A_{in}b_n}{det A}
\end{gather*}

Заметим, что числитель последнего выражения это определитель матрицы.
\begin{gather*}
  x_i = \frac{1}{det A}
  \begin{vmatrix} 
    a_{11} & a_{12} & \ldots & a_{1,i-1} & b_1 & a_{1,i+1} & \ldots & a_{1n} \\
    a_{21} & a_{22} & \ldots & a_{2,i-1} & b_2 & a_{2,i+1} & \ldots & a_{2n} \\
    \ldots & \ldots & \ldots & \ldots & \ldots & \ldots & \ldots & \ldots \\
    a_{n1} & a_{n2} & \ldots & a_{n,i-1} & b_n & a_{n,i+1} & \ldots & a_{nn} \\
  \end{vmatrix} 
\end{gather*}

\begin{note}
  Определитель $\Delta_i$ получается, если элементы i-ного столбца заменить на свободные члены СЛАУ.
\end{note}

Если квадратная матрица невырожденная, то однородная СЛАУ имеет \textit{единственное решение}. 

Если квадратная матрица вырожденная, то однородная СЛАУ имеет \textit{бесконечное количество решений}. 

\begin{theorem}
  \textit{Критерий Кронекера-Капелли}. \\
  Для того, чтобы СЛАУ была совместной, необходимо и достаточно, чтобы ранг матрицы $A$ был равен рангу матрицы A|B.
\end{theorem}
\begin{proof}
  1) Необходимость. \\
  Пусть: СЛАУ совместна, $Rg(a) = r$\\
  Базисный минор  $r \times r$:
  \begin{gather*}
    M =
    \begin{vmatrix} 
      a_{11} & a_{12} & \ldots & a_{1r} \\ 
      a_{21} & a_{22} & \ldots & a_{2r} \\ 
      \ldots & \ldots & \ldots & \ldots \\
      a_{r1} & a_{r2} & \ldots & a_{rr} \\ 
    \end{vmatrix} 
  \end{gather*}
  Если использовать векторную форму записи, то если СЛАУ имеет решение $x_1, x_2, \ldots x_{n}$, то можно записать её в виде: \[
      a_1x_1 + a_2x_2 + \ldots + a_r x_r + a_{r+1} x_{r+1} + \ldots + a_{n} x_{n} = b \tag{1}
  \]
  Согласно теореме о базисном миноре, любой столбец матрицы, который не входит в базисный минор, можно представить в виде линейной комбинации базисных столбцов:
  \begin{gather*}
    \begin{cases}
      a_{r+1} = \lambda_{1,r+1} a_1 + \lambda_{2,r+1} a_2 + \ldots + \lambda_{r,r+1} a_r \\
      a_{r+2} = \lambda_{2,r+2} a_1 + \lambda_{2,r+2} a_2 + \ldots + \lambda_{r,r+2} a_r \\
      \ldots \\
      a_{n} = \lambda_{2,n} a_1 + \lambda_{2,n} a_2 + \ldots + \lambda_{r,n} a_r
    \end{cases} \tag{2} 
  \end{gather*}
  Подставим (2) в (1):
  \begin{align*}
    a_1 x_1 + &a_2 x_2 + \ldots + a_r x_r + \\
            &+ (\lambda_{1,r+1} a_1 + \lambda_{2,r+1} a_2 \ldots + \lambda_{r,r+1} a_r) x_{r+1} + \\
            &+ \ldots + \\
            &+ (\lambda_{1,n} a_1 + \lambda_{2,n} a_2 \ldots + \lambda_{r,n} a_r) x_{n} = b
  \end{align*}
  \begin{align*}
    (x_1 + &\lambda_{1,r+1} x_{r+1} + \ldots + \lambda_{1n} x_{n}) a_1 + \\
          &+ (x_2 + \lambda_{2,r+1} x_{r+1} + \ldots + \lambda_{2n}) a_2 + \\
          &+ \ldots + \\
          &+ (x_r + \lambda_{r,r+1} x_{r+1} + \ldots + \lambda_{rn}) a_r = b \\
          &\text{где } b = const, i=1 \ldots r
  \end{align*}
  В результате столбец свободных членов можно представить в виде линейной комбинации столбцов базисного минора. Отсюда следует, что базисный минор матрицы $A$ и будет базисным минором расширенной матрицы $A|B$.
  
  Т.к $M \neq 0$ и любой окаймляющий минор $M' = 0$, то мы получаем: \[
  Rg(A) = Rg(A|B)
  \] 

  2) Достаточность. \\
  Пусть: $Rg(A) = Rg(A|B) = r$, базисный минор $M$ будет содержать первые $r$ строк и первые  $r$ столбцов базисного минора $M$.
  
  Тогда столбец  $B$ можно представить в виде линейной комбинации столбцов базисного минора $M$:
  \begin{align*}
    &b = x_1^0 a_1 + x_2^0 a_2 + \ldots + x_r^0 a_r + 0 \cdot a_{r+1} + \ldots + 0 \cdot a_n \\
    &x_1^0, x_2^0, \ldots x_r^0 \text{ -- коеффициенты линейной комбинации} \\
    &x_i^0 = const, i=1..r
  \end{align*}
  Поэтому $X = \left( x_1^0, x_2^0, \ldots x_r^0 \right) $ является решением $AX=B$, т.е. СЛАУ совместимая.
\end{proof}

