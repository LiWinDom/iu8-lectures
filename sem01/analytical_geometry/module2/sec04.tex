\subsection{Однородные СЛАУ}

Однородные СЛАУ можно записыватть в матричном виде следующим образом:
\begin{gather*}
  A \times X = \theta \\
  \theta_{m \times 1} = 
  \begin{pmatrix}
    0 \\ 0 \\ \ldots \\ 0
  \end{pmatrix}
  \quad
  X_{n \times 1} = 
  \begin{pmatrix}
    x_1 \\ x_2 \\ \ldots \\ x_{n}
  \end{pmatrix}
  \quad
  A_{m \times n} = 
  \begin{pmatrix}
    a_{11} & a_{12} & \ldots & a_{1n} \\
    a_{21} & a_{22} & \ldots & a_{2n} \\
    \ldots & \ldots & \ldots & \ldots \\
    a_{m1} & a_{m2} & \ldots & a_{mn}
  \end{pmatrix}
\end{gather*}

\begin{theorem}
  \textit{О свойствах решения однородных СЛАУ}. \\
  Пусть $X^{(1)}, X^{(2)}, \ldots X^{(n)}$ -- решения однородных СЛАУ $A \times X = \theta$. Тогда их линейной комбинацией так же является решением однородной СЛАУ.
\end{theorem}
\begin{proof}
  \begin{align*}
    X &= \lambda_1 X^{(1)} + \lambda_2 X^{(2)} + \ldots + \lambda_k X^{(k)} \\
      &= \sum_{i=1}^{k} \lambda_i X^{(i)}, \lambda_i = const
  \end{align*}
  \begin{align*}
    A \times X &= A \cdot \sum_{i=1}^{k}\lambda_i X^{(i)} = \sum_{i=1}^{k} A \lambda_i X^{(i)} = \\
          &= \sum_{i=1}^{k} \lambda_i \cdot A \times X^{(i)} = \sum_{i=1}^{k} \lambda_i \cdot \theta = \theta \\
  \end{align*}
\end{proof}

\begin{definition}
  Набор решений однородной СЛАУ называется \textit{фундаментальной системой решений (ФСР)}.\[
  k = n - r \quad r = Rg(A), n \text{ -- кол-во неизвестных СЛАУ}
  \] 
\end{definition}

\begin{theorem}
  \textit{О существовании ФСР однородной СЛАУ}. \\
  Пусть имеется однородная СЛАУ $A \times X = \theta$ с $n$ неизвестных и $rg(A) = r$. \\
  Тогда существует набор  $k = n - r$ решений однородной СЛАУ, которые образуют ФСР:  \[
    X^{(1)}, X^{(2)}, \ldots X^{(k)}
  \] 
\end{theorem}
\begin{proof}
  Пусть базисный минор $M$ матрицы A состоит из первых $r$ строк и первых $r$ столбцов матрицы $A$.
  Тогда любая строка $A$, от $r+1$ до $m$ будет линейной комбинацией строк базисного минора.

  Если  $x_1, x_2, \ldots x_{n}$ удовлетворяют уравнениям СЛАУ соответветствующим строкам базисного минора то это решение будет удовлетворять и остальным уравнениям СЛАУ.
  Поэтому исключим из системы уравнения после $r$-ой строки:
  \begin{gather*}
    \begin{cases}
      a_{11} x_1 + a_{12} x_2 + \ldots + a_{1r} x_r + \ldots + a_{1n} x_{n} = 0 \\
      a_{21} x_1 + a_{22} x_2 + \ldots + a_{2r} x_r + \ldots + a_{2n} x_{n} = 0 \\
      \ldots \\
      a_{r1} x_1 + a_{r2} x_2 + \ldots + a_{rr} x_r + \ldots + a_{rn} x_{n} = 0 \\
    \end{cases} \tag{3} 
  \end{gather*}

  Переменные, соответствующие базисным столбцам, называют \textit{базисными}, остальные -- \textit{свободными}.

  В системе (3) базисными переменными являются переменные $x_1, x_2, \ldots x_r$; свободными являются переменные $x_{r+1}, x_{r+2}, \ldots x_n$.

  Оставим в левой части слагаемые с базисными переменными, а в правой -- со свободными:
  \begin{gather*}
    \begin{cases}
      a_{11} x_1 + a_{12} x_2 + \ldots a_{1r} x_r = a_{1,r+1} x_{r+1} + \ldots + a_{1n} x_{n} \\ 
      a_{21} x_1 + a_{22} x_2 + \ldots a_{2r} x_r = a_{2,r+1} x_{r+1} + \ldots + a_{2n} x_{n} \\ 
      \ldots \\
      a_{r1} x_1 + a_{r2} x_2 + \ldots a_{rr} x_r = a_{r,r+1} x_{r+1} + \ldots + a_{rn} x_{n} \\ 
    \end{cases} \tag{4} 
  \end{gather*}

  Если свободным переменным придавать различные значения, то определитель левой части (4) равен базисному минору $A (\neq 0)$, то (4) будет иметь единственное решение.

  Возьмём $k$ наборов свободных переменных:
   \begin{gather*}
    \begin{matrix}
      X^{(1)}_{r+1} & X^{(2)}_{r+1} & \ldots & X^{(k)}_{r+1} \\
      X^{(1)}_{r+2} & X^{(2)}_{r+2} & \ldots & X^{(k)}_{r+2} \\
      \ldots & \ldots & \ldots & \ldots \\
      X^{(1)}_n = 0 & X^{(2)}_n = 0 & \ldots & X^{(k)}_n = 0 & 
    \end{matrix}
  \end{gather*}

  В результате, при каждом наборе свободных переменных мы получаем $k$ решений однородной СЛАУ:
   \begin{gather*}
     X^{(i)} = 
     \begin{pmatrix}
       X^{(i)}_1 \\
       X^{(i)}_2 \\
       \ldots \\
       X^{(i)}_r \\
       \ldots \\
       X^{(i)}_n
     \end{pmatrix}
  \end{gather*}

  Пусть линейная комбинация решений равна 0:
  \begin{gather*}
    \lambda_1
   \begin{pmatrix}
       X^{(1)}_1 \\
       X^{(1)}_2 \\
       \ldots \\
       X^{(1)}_r \\
       \ldots \\
       X^{(1)}_n
     \end{pmatrix}
     + \lambda_2
   \begin{pmatrix}
       X^{(2)}_1 \\
       X^{(2)}_2 \\
       \ldots \\
       X^{(2)}_r \\
       \ldots \\
       X^{(2)}_n
    \end{pmatrix}
    + \ldots + \lambda_k
   \begin{pmatrix}
       X^{(k)}_1 \\
       X^{(k)}_2 \\
       \ldots \\
       X^{(k)}_r \\
       \ldots \\
       X^{(k)}_n
    \end{pmatrix}
  = 
  \begin{pmatrix}
    0 \\ 0 \\ \ldots \\ 0 \\ \ldots \\ 0
  \end{pmatrix} = \theta
  \end{gather*}
  
  \begin{align*}
    r+1: \quad &1 \cdot\lambda_1 + 0 \cdot \lambda_2 + \ldots + 0 \cdot \lambda_k = 0 \implies \lambda_1 = 0 \\
    r+2: \quad &0 \cdot\lambda_1 + 1 \cdot \lambda_2 + \ldots + 0 \cdot \lambda_k = 0 \implies \lambda_2 = 0 \\
          &\ldots \\
     r: \quad &1 \lambda_1 + 0 \cdot \lambda_2 + \ldots + 1 \cdot \lambda_k = 0 \implies \lambda_k = 0
  \end{align*}

  Все коеффициенты равны нулю. Мы получили тривиальную равную нуля линейную комбинация решений однородной СЛАУ.
\end{proof}

\begin{definition}
  Если в каждом столбце ФСР все свободные переменные равны нулю, кроме одного, равного единице, то такая ФСР называется \textit{нормальной} 
\end{definition}

\begin{theorem}
  \textit{О структуре общего решения однородной СЛАУ}. \\
  Пусть $X^{(1)}, X^{(2)}, \ldots X^{(k)}$ -- ФСР некоторой СЛАУ $A \times X = \theta$. Тогда общее решение однородной СЛАУ будет иметь вид: \[
    X = c_1 X^{(1)} + c_2 X^{(2)} + \ldots + c_k X^{(k)}, \quad c_i = const
  \] 
\end{theorem}
\begin{proof}
  Пусть дана однородная СЛАУ:
  \begin{align*}
    \begin{cases}
      a_{11} x_1 + a_{12} x_2 + \ldots + a_{1n} x_{n} = 0 \\
      a_{21} x_1 + a_{22} x_2 + \ldots + a_{2n} x_{n} = 0 \\
      \ldots \\
      a_{m1} x_1 + a_{m2} x_2 + \ldots + a_{mn} x_{n} = 0
    \end{cases} \tag{1}
  \end{align*}
  Пусть $X = \left( \begin{matrix} x_1 \\ x_2 \\ \ldots \\ x_{n} \end{matrix} \right) $ -- решение СЛАУ, и матрица A имеет ранг $rg A = r$. 
  Тогда если $X$ является решением, то он ялвяется решением первых $r$ уравнений, соответствующих базисным строкам матрицы $A$.
  Пусть базисный минор стоит из первых $r$ строк и первых $r$ столбцов данной матрицы, тогда если $X$ -- решение уравнений с нулевого по $r$, то он является решением уравнений с  $r+1$ по $m$, которые являются линейной комбинацией первых  $k$ уравнений, поэтому уравнения с  $r+1$ по $m$ можно исключить.
  Т.к. базисный минор включает первые $r$ столбцов матрицы  $A$:
  \begin{gather*}
    M_r =
    \begin{pmatrix}
      a_{11} x_1 & a_{12} x_2 & \ldots & a_{1r} x_r \\
      a_{21} x_1 & a_{22} x_2 & \ldots & a_{2r} x_r \\
      \ldots & \ldots & \ldots & \ldots \\
      a_{r1} x_1 & a_{r2} x_2 & \ldots & a_{rr} x_r \\
      \ldots & \ldots & \ldots & \ldots \\
      a_{m1} x_1 & a_{m2} x_2 & \ldots & a_{mr} x_r \\
    \end{pmatrix}
  \end{gather*}
  то соответствующие этим столбцам переменные являются базисными (с $x_1$ по $x_{r}$), а остальные переменные (c $x_{r+1}$ по $ x_n$ ) -- свободными.

  После исключения первых $r$ строк, получаем:
  \begin{gather*}
    \begin{cases}
      a_{r1} x_1 + a_{r2} x_2 + \ldots + a_{rn} x_{n} = 0 \\
      a_{r+1,1} x_1 + a_{r+1,2} x_2 + \ldots + a_{r+1,n} x_{n} = 0 \\
      \ldots \\
      a_{m1} x_1 + a_{m2} x_2 + \ldots + a_{mr} x_{r} + \ldots + a_{mn} x_{n} = 0
    \end{cases} \tag{2}
  \end{gather*}

  Преобразуем уравнения так, что в левой части остались базисные переменные, а в правой -- свободные:
  \begin{gather*}
    \begin{cases}
      a_{11} x_1 + a_{12} + x_2 + \ldots + a_{1r} x_{r} = a_{1r+1} x_{r+1} + \ldots + a_{1n} x_{n} = 0 \\
      a_{21} x_1 + a_{22} + x_2 + \ldots + a_{2r} x_{r} = a_{2r+1} x_{r+1} + \ldots + a_{2n} x_{n} = 0 \\
    \ldots \\
      a_{m1} x_1 + a_{m2} + x_2 + \ldots + a_{mr} x_{r} = a_{mr+1} x_{r+1} + \ldots + a_{mn} x_{n} = 0 \tag{3}
    \end{cases}
  \end{gather*}
  Задавая различные значения свободных переменных, мы получаем, что система (3) будет иметь единственное решение, т.к. главный определитель данной системы будет равен угловому минору, не равному нулю.
  Решая эту систему получаем решение:
  \begin{gather*}
    \begin{cases}
      x_1 = x_{r+1} \lambda_{1,r+1} + x_{r+2} \lambda_{1,r+2} + \ldots + x_{n} \lambda_{1, n} \\
      x_2 = x_{r+1} \lambda_{2,r+1} + x_{r+2} \lambda_{2,r+2} + \ldots + x_{n} \lambda_{2, n} \\
      \ldots \\
      x_r = x_{r+1} \lambda_{r,r+1} + x_{r+2} \lambda_{r,r+2} + \ldots + x_{n} \lambda_{r, n} \tag{4}
    \end{cases}
  \end{gather*}
  Т.к. $X^{(1)}, X^{(2)}, \ldots X^{(k)}$ образуют ФСР, то они удовлетворяют системе (4):
  \begin{gather*}
    \begin{cases}
      X_1^{(i)} = X_{r+1}^{(i)} \lambda_{1,r+1} + X_{r+2}^{(i)} \lambda_{1,r+2} + \ldots + X_{n}^{(i)} \lambda_{1, n} \\
      X_2^{(i)} = X_{r+1}^{(i)} \lambda_{2,r+1} + X_{r+2}^{(i)} \lambda_{2,r+2} + \ldots + X_{n}^{(i)} \lambda_{2, n} \\
      \ldots \\
      X_r^{(i)} = X_{r+1}^{(i)} \lambda_{n,r+1} + X_{r+2}^{(i)} \lambda_{n,r+2} + \ldots + X_{n}^{(i)} \lambda_{n, n} \tag{5}
    \end{cases} \quad 
    X^{(i)} = 
    \begin{pmatrix}
      x_1^{(i)} \\ x_2^{(i)} \\ \ldots \\ x_m^{(i)}
    \end{pmatrix}
  \end{gather*}

  Составим матрицу $B$ из столбцов $X^{(i)}$ :
  \begin{align*}
    B =
    \begin{pmatrix}
      x_1 & x_1^{(1)} & x_1^{(2)} & \ldots & x_1^{(n)} \\
      x_2 & x_2^{(1)} & x_2^{(2)} & \ldots & x_2^{(n)} \\
      \ldots & \ldots & \ldots & \ldots & \ldots \\
      x_r & x_r^{(1)} & x_r^{(2)} & \ldots & x_r^{(n)} \\
      \ldots & \ldots & \ldots & \ldots & \ldots \\
      x_n & x_n^{(1)} & x_n^{(2)} & \ldots & x_n^{(n)} \\
    \end{pmatrix} \\
    \begin{matrix}
      X & X^{(1)} & X^{(2)} & \ldots & X^{(n)}
    \end{matrix}
  \end{align*} 

  Вычтем из элементов первой строки соответствующие элементы строк с $r+1$ по $m$ с соответствующим коеффициентом  $\lambda_{1,r+1}, \lambda_{1,r+2}, \ldots \lambda_{1n}$:
  \begin{gather*}
    \begin{cases}
      x_1^{(1)} - \lambda_{1,r+1} x_{r+1}^{(1)} - \lambda_{1,r+2} x_{r+2}^{(2)} - \ldots - \lambda_{1,n} x_{n}^{(1)} = 0 \\
      x_1^{(2)} - \lambda_{1,r+1} x_{r+1}^{(2)} - \lambda_{1,r+2} x_{r+2}^{(2)} - \ldots - \lambda_{1,n} x_{n}^{(2)} = 0 \\
      \ldots \\
      x_1^{(k)} - \lambda_{1,r+1} x_{r+1}^{(k)} - \lambda_{1,r+2} x_{r+2}^{(k)} - \ldots - \lambda_{1,n} x_{n}^{(k)} = 0 \\
      x_1 - \lambda_{1,r+1} x_{r+1} - \lambda_{1,r+2} x_{r+2} - \ldots - \lambda_{1,n} x_{n} = 0
    \end{cases}
  \end{gather*}

  Аналогично вычитая из строк до r строки $r+1$ до $n$ с коеффициентами  $\lambda$.
  В результате получаем, что в преобразованной матрице $B$ первые  $r$ строк будут нулевые:
  
  \begin{gather*}
    B =
    \begin{pmatrix}
      0 & 0 & 0 & \ldots & 0 \\
      0 & 0 & 0 & \ldots & 0 \\
      \ldots & \ldots & \ldots & \ldots & \ldots \\
      0 & 0 & 0 & \ldots & 0 \\
      x_{r+1} & x_{r+1}^{(1)} & x_{r+1}^{(2)} & \ldots & x_{r+1}^{(n)} \\
      \ldots & \ldots & \ldots & \ldots & \ldots \\
      x_n & x_n^{(1)} & x_n^{(2)} & \ldots & x_n^{(n)} \\
    \end{pmatrix}
  \end{gather*} 

  Поскольку элементарные преобразования не меняют ранга матрицы, получаем, что ранг матрицы $B$ будет равен $k = n - r$.
  Так как по условию столбцы $X^{(1)}, X^{(2)}, \ldots X^{(k)}$ образуют ФСР, они являются линейно-независимыми. Поэтому первый столбец можно представить в виде линейной комбинации столбцов.
\end{proof}

\subsection{Неоднородные СЛАУ}

\begin{theorem}
  \textit{О связи решений неоднородной и соответствующей однородной СЛАУ}. \\
  Пусть $X^{(0)}$ -- некоторое решение неоднордной СЛАУ $A \times X = B$. Произвольный стобец $X$ является решением СЛАУ $A \times X = B$ тогда и только тогда, когда его можно представить в виде:  \[
    X = X^{(0)} + Y, \quad \text{ где } A \times Y = \theta
  \] 
\end{theorem}
\begin{proof}
  1) Необходимость. \\
  Пусть $X$ -- решение СЛАУ $A \times X = B$. Обозначим $Y = X - X^{(0)}$
  \begin{gather*}
    A \times Y = A \times (X - X^{(0)}) = A \times X - A \times X^{(0)} = B - B = \theta
  \end{gather*}
  Значит $Y$ является решением соответствующей однородной СЛАУ  $A \times Y = \theta$.
  
  2) Достаточность. \\
  Пусть $X$ можно представить в виде:  \[
    X = X^{(0)} + Y, \quad \text{ где } A \times Y = 0
  \]
  Тогда:
  \begin{gather*}
  A \times X = A \times (X^{(0)} + Y) = A \times X^{(0) + A \times Y} = B + \theta = B
  \end{gather*}
  Отсюда делаем вывод, что $X$ является решением неоднородной СЛАУ.
\end{proof}

\begin{theorem}
  \textit{О структуре общего решения неоднородной СЛАУ}. \\ 
  Пусть $X^{(0)}$ -- некоторое частное решение неоднородной СЛАУ  $A \times X = B$.
  Пусть  $X^{(1)} \ldots$ -- некоторая ФСР, соответствующая однородной СЛАУ $A \times X=\theta$.
  Тогда общее решение неоднородной СЛАУ $A \times X = B$ будет иметь вид :  \[
    X_o = X^{(0)} + c_1 X^{(1)} + c_2 X^{(2)} + \ldots + c_n X^{(n)}, \quad c_i = const
  \] 
\end{theorem}

