\section{Прямая в пространстве}

\subsection{Способы задания прямой в пространстве}

\subsubsection{Каноническое уравнение прямой}

Пусть прямая $l$ проходит через точку $M_0\left(x_0, y_0, z_0 \right)$ и имеет направляющий вектор $\vec{S} = \{m, n, p\}$.
Возьмём на прямой $l$ произвольную точку $M(x, y, z)$.
Составим вектор: \[
\overrightarrow{M_0M} = \{x - x_0, y - y_0, z - z_0\} 
\] 
Отсюда:
\begin{gather*}
  \overrightarrow{M_0M} \parallel \vec{S} \implies \boxed{\frac{x - x_0}{m} = \frac{y - y_0}{n} = \frac{z - z_0}{p}}
\end{gather*}

\subsubsection{Параметрическое уравнение}

Пусть прямая $l$ задана каноническим уравнением:  \[
l: \frac{x - x_0}{m} = \frac{y - y_0}{n} = \frac{z - z_0}{p} = (t)
\] 
Отсюда:
\begin{gather*}
  \begin{cases}
  \frac{x - x_0}{m} = t \\
  \frac{y - y_0}{n} = t \\
  \frac{z - z_0}{p} = t \\
  \end{cases} \implies \boxed{
  \begin{cases}
    x = x_0 + mt \\
    y = y_0 + nt \\
    z = z_0 + pt \\
  \end{cases}}
\end{gather*}

\subsubsection{Через две точки}

Пусть прямая $l$ проходит через точки $M_0(x_0, y_0, z_0)$ и $M_1(x_1, y_1, z_1)$.
Возьмём на прямой $l$ точку  $M(x, y, z)$.
Составим два вектора:
\begin{gather*}
  \overrightarrow{M_0M} = \{x - x_0, y - y_0, z - z_0\} \\ 
  \overrightarrow{M_0M_1} = \{x_1 - x_0, y_1 - y_0, z_1 - z_0\}
\end{gather*}
Отсюда:
\begin{gather*}
  \overrightarrow{M_0M} \parallel  \overrightarrow{M_0M_1} \implies \\
  \boxed{
    \frac{x - x_0}{x_1 - x_0} = \frac{y - y_0}{y_1 - y_0} = \frac{z - z_0}{z_1 - z_0}
  }
\end{gather*}

\subsubsection{Общее уравнение}

Пусть плоскости $\alpha_1$ и $\alpha_2$ заданы общими уравнениями:
\begin{gather*}
  \alpha_1: A_1x + B_1y + C_1z + D_1 = 0 \\
  \alpha_2: A_2x + B_2y + C_2z + D_2 = 0 \\
\end{gather*}
Если $\alpha_1 \not \parallel \alpha_2$, то они пересекаются по прямой $l$.
Тогда $\forall M(x, y, z) \in l$ будет выполнятся система:
\begin{gather*}
  \boxed{
    \begin{cases}
      A_1x + B_1y + C_1z + D_1 = 0 \\
      A_2x + B_2y + C_2z + D_2 = 0 \\
    \end{cases}
  }
\end{gather*}

\begin{eg}
  Составить уравнение прямой, являющейся пересечением плоскостей:
  \begin{gather*}
    \alpha_1: 2x + y - z + 4 = 0 \\
    \alpha_2: 3x + 2y + z - 6 = 0
  \end{gather*}

Для того, чтобы содать уравнение прямой $l$, нужно знать  $M_0(x_0, y_0, z_0)$ направляющий вектор $\vec{S} = \{m, n, p\}$.

\begin{gather*}
  \text{Из (1) } \implies \vec{n_1} = \{2, 1, -1\} \\ 
  \text{Из (2) } \implies \vec{n_2} = \{3, 2, 1\} \\
  \alpha_1 \neq \alpha_2
\end{gather*}

Найдем точку $M_0$.
Пусть $z_0 = 0$ (прямая обязательно пересечёт плоскость oXY):

\begin{gather*}
  \begin{cases}
    2x_0 + y_0 + 4 = 0 \\
    3x_0 + 2y_0 - 6 = 0
  \end{cases} \implies 
  \begin{cases}
    4x_0 + 2y_0 + 8 = 0 \\
    3x_0 + 2y_0 + - 6 = 0
  \end{cases} \implies
  \begin{cases}
    x_0 = -14 \\
    y_0 = 24
  \end{cases} \\
  \\
  \implies M_0(-14, 24, 0)
\end{gather*}

Найдем направляющий вектор $\vec{S}$
\begin{align*}
  \vec{s} &= \vec{n_2} \cdot \vec{n_1} \\
          &= 
  \begin{vmatrix}
    \vec{i} & \vec{j} & \vec{k} \\
    3 & 2 & 1 \\
    2 & 1 & -1
  \end{vmatrix} = -3 \vec{i} + 5 \vec{j} - \vec{k}
\end{align*}
\begin{gather*}
  \implies \vec{S} = \{-3, 5, 1\} 
\end{gather*}

Составим каноническое уравнение прямой:
\begin{gather*}
  \frac{x + 14}{-3} = \frac{y - 24}{5} = \frac{z}{-1}
\end{gather*}
\end{eg}

\subsection{Расстояние от точки до прямой в пространстве}

Пусть прямая $l$ задана каноническим уравнением:  \[
l: \frac{x - x_0}{m} = \frac{y - y_0}{n} = \frac{z - z_0}{p}
\] \[
\vec{S} = \{m, n, p\} 
\] 
Задана точка $M_1(x_1, y_1, z_1) \not \in l$.
Составим вектор:  \[
\overrightarrow{M_0M_1} = \{x_1 - x_0, y_1 - y_0, z_1 - z_0\} 
\] 
Построим на векторах $\vec{S}$ и $\overrightarrow{M_0M}$ параллелограмм.
Тогда высота этого параллелограмма из точки $M_1$ и есть искомое расстояние от точки $M_1$ до прямой $l$.
\begin{gather*}
  S_{par} = h \cdot |\vec{S}| \\
  h = \frac{S_{par}}{|\vec{S}|} \\
  S_{par} = |\overrightarrow{M_0M_1} \times \vec{S}| 
\end{gather*}
Тогда:
\begin{gather*}
  \overrightarrow{M_0M_1} \times \vec{S} = 
  \begin{vmatrix}
    \vec{i} & \vec{j} & \vec{k} \\
    x_1 - x_0 & y_1 - y_0 & z_1 - z_0 \\
    m & n & p
  \end{vmatrix} = \\
  = \begin{vmatrix}
    y_1 - y_0 & z_1 - z_0 \\
    n & p
  \end{vmatrix} \cdot \vec{i} -
  \begin{vmatrix}
    x_1 - x_0 & z_1 - z_0 \\
    m & p
  \end{vmatrix} \cdot \vec{j} + 
  \begin{vmatrix}
    x_1 - x_0 & y_1 - y_0 \\
    m & n
  \end{vmatrix} \cdot \vec{k} \implies \\
  \\
  \overrightarrow{M_0M_1 \times \vec{S}} = 
  \left\{
  \begin{vmatrix}
    y_1 - y_0 & z_1 - z_0 \\
    n & p
  \end{vmatrix},
   - \begin{vmatrix}
    x_1 - x_0 & z_1 - z_0 \\
    m & p
  \end{vmatrix}, 
  \begin{vmatrix}
    x_1 - x_0 & y_1 - y_0 \\
    m & n
  \end{vmatrix}
  \right\} \implies \\
  \\
  |\overrightarrow{M_0M_1 \times \vec{S}}| = 
  \sqrt{
  \begin{vmatrix}
    y_1 - y_0 & z_1 - z_0 \\
    n & p
  \end{vmatrix}^2 +
  \begin{vmatrix}
    x_1 - x_0 & z_1 - z_0 \\
    m & p
  \end{vmatrix}^2 + 
  \begin{vmatrix}
    x_1 - x_0 & y_1 - y_0 \\
    m & n
  \end{vmatrix}^2
  } \implies \\
  \rho(M_1, l) = 
  \frac{|\overrightarrow{M_1M_0} \times \vec{S}|}{|\vec{S}|} = \\
  \boxed{\frac{\sqrt{
  \begin{vmatrix}
    y_1 - y_0 & z_1 - z_0 \\
    n & p
  \end{vmatrix}^2 +
  \begin{vmatrix}
    x_1 - x_0 & z_1 - z_0 \\
    m & p
  \end{vmatrix}^2 +
  \begin{vmatrix}
    x_1 - x_0 & y_1 - y_0 \\
    m & n
  \end{vmatrix}^2
  }}{\sqrt{m^2 + n^2 + p^2} }} 
\end{gather*}

\subsubsection{Расстояние между параллельными прямыми}

Пусть прямые заданы каноническими уравнениями:
\begin{gather*}
  l_1: \frac{x - x_0}{m_1} = \frac{y - y_0}{n_1} = \frac{z - z_0}{p_1} \implies M_0(x_0, y_0, z_0), \vec{S_1} = \{m_1, n_1, p_1\} \\
  l_2: \frac{x - x_0}{m_2} = \frac{y - y_0}{n_2} = \frac{z - z_0}{p_2} \implies M_0(x_0, y_0, z_0), \vec{S_2} = \{m_2, n_2, p_2\} \\
  l_1 \parallel l_2 \implies \frac{m_1}{m_2} = \frac{n_1}{n_2} = \frac{p_1}{p_2}
\end{gather*}

Построим параллелограмм на векторах $\overrightarrow{S_1}$ и $\overrightarrow{M_1M_2}$.
Тогда расстояние между прямыми $l_1$ и $l_2$ будет высота данного параллелограмма.
\begin{gather*}
  \rho(l_1, l_2) = \frac{|\overrightarrow{M_1M_0} \times \vec{S}|}{|\vec{S}|} = \\
  \boxed{\frac{\sqrt{
  \begin{vmatrix}
    y_2 - y_1 & z_2 - z_1 \\
    n_1 & p_1
  \end{vmatrix}^2,
  \begin{vmatrix}
    x_2 - x_1 & z_2 - z_1 \\
    m_1 & p_1
  \end{vmatrix}^2, 
  \begin{vmatrix}
    x_2 - x_1 & y_2 - y_1 \\
    m_1 & n_1
  \end{vmatrix}^2
}}{\sqrt{m_1^2 + n_1^2 + p_1^2} }}
\end{gather*}

\subsubsection{Расстояние между скрещивающимися прямыми}

Пусть прямые заданы каноническими уравнениями:
\begin{gather*}
  l_1: \frac{x - x_1}{m_1} = \frac{y - y_1}{n_1} = \frac{z - z_1}{p_1} \implies M_1(x_1, y_1, z_1), \quad \vec{S_1} = \{m_1, n_1, p_1\} \\
  l_2: \frac{x - x_2}{m_2} = \frac{y - y_2}{n_2} = \frac{z - z_2}{p_2} \implies M_2(x_2, y_2, z_2), \quad \vec{S_2} = \{m_2, n_2, p_2\}
\end{gather*}

Составим вектор $\overrightarrow{M_1M_2}$: \[
\overrightarrow{M_1M_2} = \{x_2 - x_1, y_2 - y_1, z_2 - z_1\} 
\] 

Вектора $\vec{S}$ и $\overrightarrow{M_1M_2}$ не компланарны. 
Поэтому на этих векторах можно построить параллелепипед.
Тогда расстояние между прямыми $l_1$ и $l_2$ будет равно высоте этого параллелепипеда.

\begin{gather*}
  V = |\overrightarrow{M_1M_2} \cdot \vec{s_1} \cdot \vec{s_2}| \\
  V= h \cdot S
\end{gather*}
\begin{gather*}
  \overrightarrow{M_1M_2} \cdot \vec{s_1} \cdot \vec{s_2} = 
  \begin{vmatrix}
    x_2 - x_1 & y_2 - y_1 & z_2 - z_1 \\
    m_1 & n_1 & p_1 \\
    m_2 & n_2 & p_2
  \end{vmatrix}
\end{gather*}
\begin{align*}
  S = |\vec{s_1} \times \vec{s_2}| &= 
  \begin{vmatrix}
    \vec{i} & \vec{j} & \vec{k} \\
    m_1 & n_1 & p_1 \\
    m_2 & n_2 & p_2 \\
  \end{vmatrix} \\
  &=
  \vec{i} \cdot 
  \begin{vmatrix}
    n_1 & p_1 \\
    n_2 & p_2
  \end{vmatrix} - \vec{j} \cdot 
  \begin{vmatrix}
    m_1 & p_1 \\
    m_2 & p_2
  \end{vmatrix} + \vec{k} \cdot 
  \begin{vmatrix}
    m_1 & n_1 \\
    m_2 & n_2
  \end{vmatrix} \\
\end{align*}
\begin{gather*}
  \implies \boxed{\rho(l_2, l_1) = 
  \frac{
    \left|
    \begin{vmatrix}
      x_2 - x_1 & y_2 - y_1 & x_2 - x_1 \\
      m_1 & n_1 & p_1 \\
      m_2 & n_2 & p_2
    \end{vmatrix}
    \right| 
  }{
  \sqrt{
    \begin{vmatrix}
      n_1 & p_1 \\
      n_2 & p_2
    \end{vmatrix}^2 +
    \begin{vmatrix}
      m_1 & p_1 \\
      m_2 & p_2
    \end{vmatrix}^2 +  
    \begin{vmatrix}
      m_1 & n_1 \\
      m_2 & n_2
    \end{vmatrix}^2 
  } 
  }}
\end{gather*}
  

\subsection{Взаимное расположение прямых в пространстве}

Пусть прямые $l_1$ и $l_2$ заданы каноническими уравнениями:
\begin{gather*}  
  l_1: \frac{x - x_1}{m_1} = \frac{y - y_1}{n_1} = \frac{z - z_1}{p_1} \implies M_1(x_1, y_1, z_1), \vec{S_1} = \{m_1, n_1, p_1\} \\
  l_2: \frac{x - x_2}{m_2} = \frac{y - y_2}{n_2} = \frac{z - z_2}{p_2} \implies M_2(x_2, y_2, z_2), \vec{S_2} = \{m_2, n_2, p_2\}
\end{gather*}

Составим вектор $\overrightarrow{M_1M_2}$.

\subsubsection{Совпадают}
Если прямые $l_1$ и $l_2$ \textbf{совпадают}, то:
\begin{gather*}
    \frac{m_1}{m_2} = \frac{n_1}{n_1} = \frac{p_1}{p_1} \\
    \frac{x_2 - x_1}{m_1} = \frac{y_2 - y_1}{n_1} = \frac{z_2 - z_1}{p_1}
\end{gather*}

\subsubsection{Параллельны}
Если прямые $l_1$ и $l_2$ \textbf{параллельны} то: 
\begin{gather*}
  \frac{m_1}{m_2} = \frac{n_1}{n_2} = \frac{p_1}{p_2} \\
\end{gather*}
\begin{center}
  И \textbf{не} выполняется условие:
\end{center}
\begin{gather*}
  \frac{x_2 - x_1}{m_1} = \frac{y_2 - y_1}{n_1} = \frac{z_2 - z_1}{p_1}
\end{gather*}

\subsubsection{Пересекаются}
Если прямые $l_1$ и $l_2$ \textbf{пересекаются}, они лежат в одной плоскости.
В таком случае вектора $\overrightarrow{M_1M_2}, \vec{s_1}, \vec{s_2}$ - компланарны:
\begin{gather*}
  \overrightarrow{M_1M_2} \cdot \vec{S_1} \cdot \vec{S_2} = 0 \\
  \boxed{
    \begin{vmatrix}
      x_2 - x_1 & y_2 - y_1 & z_2 - z_1 \\
      m_1 & n_1 & p_1 \\
      m_2 & n_2 & p_2
    \end{vmatrix} = 0
  }
\end{gather*}

\subsubsection{Скрещиваются}
Если прямые $l_1$ и $l_2$ \textbf{скрещиваются}, то они не лежат в одной плоскости.
В таком случае вектора $\overrightarrow{M_1M_2}, \vec{s_1}, \vec{s_2}$ - некомпланарны:
\begin{gather*}
  \overrightarrow{M_1M_2} \cdot \vec{S_1} \cdot \vec{S_2} \neq 0 \\
  \boxed{
    \begin{vmatrix}
      x_2 - x_1 & y_2 - y_1 & z_2 - z_1 \\
      m_1 & n_1 & p_1 \\
      m_2 & n_2 & p_2
    \end{vmatrix} \neq 0
  }
\end{gather*}


\subsection{Угол между прямой и плоскостью}

Пусть плоскость задана общим уравнением: \[
\alpha: Ax + By + Cz + D = 0
\] 

Пусть прямая $l$ задана каноническим уравнением:  \[
  l: \frac{x - x_0}{m} = \frac{y - y_0}{n} = \frac{z - z_0}{p} \qquad \vec{s} = \{m, n, p\} 
\] 
Обозначим угол $\varphi$ - между прямой плоскостью, и $\beta$ - между прямой и нормалью.
Тогда:
\begin{gather*}
  \cos \beta = \frac{\vec{n} \cdot \vec{s}}{|\vec{n}|\cdot|\vec{s}|} \\
  \beta = 90 - \varphi \\
  \cos(90 - \varphi) = \frac{\vec{n} \cdot \vec{s}}{|\vec{n}|\cdot|\vec{s}|} \\
  \sin(\varphi) = \frac{\vec{n} \cdot \vec{s}}{|\vec{n}|\cdot|\vec{s}|} \\
  \boxed{\sin \varphi = \frac{|Am + Bn + Cp|}{\sqrt{A^2 + B^2 + C^2} \cdot \sqrt{m^2 + n^2 + p^2} }}
\end{gather*}

\subsubsection{Условие параллельности прямой и плоскости}

Пусть плоскость задана общим уравнением: \[
  \alpha: Ax + By + Cz + D = 0 \qquad \vec{n} = \{A, B, C\} 
\] 

Пусть прямая $l$ задана каноническим уравнением:  \[
  l: \frac{x - x_0}{m} = \frac{y - y_0}{n} = \frac{z - z_0}{p} \qquad \vec{s} = \{m, n, p\} 
\] 
\begin{gather*}
  l \parallel \alpha \implies \vec{n} \perp \vec{s} \implies \vec{n} \cdot \vec{s} = 0 \\
  \boxed{Am + Bn + Cp = 0}
\end{gather*}

\subsubsection{Условие перпендикулярности прямой и плоскости}

Пусть плоскость задана общим уравнением: \[
  \alpha: Ax + By + Cz + D = 0 \qquad \vec{n} = \{A, B, C\} 
\] 

Пусть прямая $l$ задана каноническим уравнением:  \[
  l: \frac{x - x_0}{m} = \frac{y - y_0}{n} = \frac{z - z_0}{p} \qquad \vec{s} = \{m, n, p\} 
\] 
\begin{gather*}
  l \perp \alpha \implies \vec{n} \parallel \vec{s} \implies
  \boxed{\frac{A}{m} = \frac{B}{n} = \frac{C}{p}}
\end{gather*}

\subsubsection{Примеры задач}

\begin{eg}
  Задача: \textit{составить уравнение прямой $l_2$ симметричной прямой $l_1$, которая задана каноническим уравнением}: \[
  \frac{x - 1}{2} = \frac{y}{1} = \frac{z + 1}{0} \qquad \vec{s} = \{2, 1, 0\} 
  \] 
  относительно плоскости $\alpha$: \[
    \alpha: x - y + 2z - 1 = 0 \qquad \vec{n} = \{1, -1, 2\} 
  \] 
  Решение:
  (1) Проверим, является ли прямая $l_1$ параллельной плоскости $\alpha$:
  \begin{gather*}
    \vec{s} \cdot \vec{n} = 2 \cdot 1 + 1 \cdot (-1) + 0 \cdot 2 = 2 - 1 = 1 \neq 0 \implies l_1 \not \parallel \alpha
  \end{gather*}

  (2) Находим точку пересечения прямой $l$ с плоскостью $\alpha$ - пусть это точка $A(x_2, y_2, z_2)$.
  Из канонического уравнения прямой $l_1$ получим параметрическое уравнение:
  \begin{gather*}
    \frac{x - 1}{2} = \frac{y}{1} = \frac{z + 1}{0} = t \\
    \begin{cases}
      \frac{x - 1}{2} = t \\
      \frac{y}{1} = t \\
      \frac{z + 1}{0} = t
    \end{cases} \implies
    \begin{cases}
      x = 2t + 1 \\
      y = t \\
      z = -1
    \end{cases}
  \end{gather*}
  Т.к. точка $A$ принадлежит и прямой, и плоскости, то её координаты удовлетворяют и параметрическому уравнению прямой, и общему уравнению плоскости:
  \begin{gather*}
    2t + 1 - t - 2 - 1 = 0 \\
    t = 2 \implies \begin{cases}
      x_2 = 5 \\
      y_2 = 2 \\
      z_2 = -1
    \end{cases} \implies A(5, 2, -1)
  \end{gather*}

  (3) Из канонического уравнения прямой возьмем точку $M_1(1, 0, -1) \in l_1$. 
  Найдем ей симметричную относительно плоскости $\alpha$ точку $M_2(x_2, y_2, z_2)$
  Составим уравнение прямой $l_3$, проходящей через точку $M_1$ и с направляющим вектором $\vec{n}$. \[
    \frac{x - 1}{1} = \frac{y}{-1} = \frac{z + 1}{2}
  \] 
  Найдём точку пересечения $O(x_3, y_3, z_3)$ прямой $l_3$ с плоскостью $\alpha$. 
  Составим параметрическое уравнение прямой $l_3$:
  \begin{gather*}
    \frac{x - 1}{1} = \frac{y}{-1} = \frac{z + 1}{2} = t \\
    \begin{cases}
      \frac{x - 1}{1} = t \\
      \frac{y}{-1} = t \\
      \frac{z + 1}{2} = t
    \end{cases} \implies
    \begin{cases}
      x = t + 1 \\
      y = -t \\
      z = 2t - 1
    \end{cases}
  \end{gather*}
  Т.к. точка $O$ принадлежит и прямой, и плоскости, то её координаты удовлетворяют и параметрическому уравнению прямой, и общему уравнению плоскости:
  \begin{gather*}
    t + 1 + t + 4t - 2 = 0 \\
    t = \frac{1}{3} \implies \begin{cases}
      x_3 = \frac{4}{3} \\
      y_3 = -\frac{1}{3} \\
      z_3 = \frac{1}{3}
    \end{cases}
  \end{gather*}

  Составляем вектор $\overrightarrow{M_1O}$: \[
  \overrightarrow{M_1O} = \left\{\frac{1}{3}, -\frac{1}{3}, \frac{2}{3}\right\} 
  \] 

  Пусть $M_2(x_2, y_2, z_2)$. Тогда:
  \begin{gather*}
    \overrightarrow{OM_2} = \{x_2, y_2, z_2\} \\
    \overrightarrow{M_1O} = \overrightarrow{OM_2} \implies \begin{cases}
      x_2 - \frac{4}{3} = \frac{1}{3} \\
      y_2 + \frac{1}{3} = -\frac{1}{3} \\
      z_2 + \frac{1}{3} = \frac{2}{3}
    \end{cases} \iff \begin{cases}
      x_2 = \frac{5}{3} \\
      y_2 = \frac{2}{3} \\
      z_2 = \frac{1}{3}
    \end{cases}
  \end{gather*}

  4) Составляем уравнение прямой, проходящей через точки $A\left(5, 2, -1\right)$ и $M_2\left(\frac{5}{3}, -\frac{2}{3}, \frac{1}{3}\right)$:
  \begin{gather*}
    \frac{x - x_a}{x_1 - x_a} = \frac{y - y_a}{y_2 - y_a} = \frac{z - z_a}{z_2 - z_a} \\
    \frac{x - 5}{\frac{5}{3} - 5} = \frac{y - 2}{-\frac{2}{3} - 2} = \frac{z + 1}{\frac{1}{3} + 1}  \\
    \frac{x - 5}{-\frac{10}{3}} = \frac{y - 2}{-\frac{8}{3}} = \frac{z + 1}{\frac{4}{3}} \\
    \boxed{\frac{x - 5}{-5} = \frac{y - 2}{-4} = \frac{z + 1}{4}}
  \end{gather*}
\end{eg}

\begin{eg}
  Задача: \textit{Составить каноническое уравнение общего перпендикуляра к прямым $l_1$ и $l_2$, заданными параметрическими уравнениями}:
  \begin{gather*}
    l_1: \begin{cases}
      x = t + 2 \\
      y = 3t + 4 \\
      z = -2t - 2
    \end{cases} \qquad
    l_2: \begin{cases}
      x = -3t + 2 \\
      y = t \\
      z = 3t - 4
    \end{cases}
  \end{gather*}
  Решение: \\
  1) Составим канонические уравнения прямых для $l_1$ и $l_2$:
  \begin{gather*}
    l_1: \begin{cases}
      t = \frac{x - 2}{2} \\
      t = \frac{y - 4}{3} \\
      t = \frac{z + 2}{-2}
    \end{cases} \implies
    \frac{x - 2}{2} = \frac{y - 4}{3} = \frac{z + 2}{-2} \implies \\
    M_1(2, 4, -2) \qquad \vec{s_1} = \{2, 3, -2\} 
  \end{gather*}
  \begin{gather*}
    l_2: \begin{cases}
      t = \frac{x - 1}{-3} \\
      t = \frac{y}{1} \\
      t = \frac{z + 4}{3}
    \end{cases} \implies
    \frac{x - 1}{-3} = \frac{y - 4}{3} = \frac{z + 2}{-2} \implies \\
    M_2(1, 4, -2) \qquad \vec{s_2} = \{-3, 1, 3\} 
  \end{gather*}

  Найдём вектор $\overrightarrow{M_1M_2}$:
  \begin{gather*}
    \overrightarrow{M_1M_2} = \{0, -4, -2\} 
  \end{gather*}

  Проверим, являются ли прямые $l_1$ и $l_2$ скрещивающимися или параллельными. Найдём смешанное произведение $\overrightarrow{M_1M_2} \vec{s_1} \vec{s_2}$:
  \begin{gather*}
    \overrightarrow{M_1M_2} \vec{s_1} \vec{s_2} = 
    \begin{vmatrix}
      0 & -4 & -2 \\
      2 & 3 & -2 \\
      -3 & 1 & 3
    \end{vmatrix} = -22 \neq 0
  \end{gather*}
  Значит, прямые не лежат в одной плоскости, следовательно, они скрещивающиеся.

  2) Найдем направляющий вектор общего перпендикуляра к прямым $l_1$ и $l_2$.
  \begin{gather*}
    \begin{rcases}
      \vec{s} \perp \vec{s_1} \\
      \vec{s} \perp \vec{s_2}
    \end{rcases}
    \iff \vec{s} = \vec{s_1} \times \vec{s_2} \\
    \vec{s_1} \times \vec{s_2} = 
    \begin{vmatrix}
      \vec{i} & \vec{j} & \vec{k} \\
      2 & 3 & -2 \\
      -3 & 1 & 3
    \end{vmatrix} = 11 \vec{i} + 11 \vec{k} \implies \vec{s} = \{1, 0, 1\} \\ 
  \end{gather*}

  3) Составим уравнение плоскости $\alpha_1$, проходящей через точки $M_1$ и вектора $\vec{s_1} \vec{s}$. 
  Возьмём произвольную точку $M(x, y, z) \in \alpha_1$.
  Составим вектор $\overrightarrow{M_1M}$: \[
  \overrightarrow{M_1M} = \{x - 2, y - 4, z + 2\} 
  \]
  Вектора $\overrightarrow{M_1M}, \vec{s_1}, \vec{s}$ - компланарные, а следовательно: 
  \begin{gather*}
    \overrightarrow{M_1M} \vec{s_1} \vec{s} = 0 \\
    \overrightarrow{M_1M} \vec{s_1} \vec{s} = 
    \begin{vmatrix}
      x - 2 & y - 4 & z + 2 \\
      2 & -3 & 2 \\
      1 & 0 & 1
    \end{vmatrix} = 
    -3x + 4y + 3z  - 4\\
    \boxed{\alpha_1: -3x + 4y + 3z - 4 = 0}
  \end{gather*}

  4) Составим плоскость $\alpha_2$ через точку $M_2$  и вектора $\vec{s_1}$ и $\vec{s_2}$.
  Возьмём произвольную точку $M(x, y, z) \in \alpha_2$.
  Составим вектор $\overrightarrow{M_2M}$: \[
  \overrightarrow{M_2M} = \{x - 2, y, z + 4\} 
  \] 
  Вектора $\overrightarrow{M_2M}, \vec{s_2}, \vec{s}$ - компланарные, а следовательно: 
  \begin{gather*}
    \overrightarrow{M_2M} \vec{s_2} \vec{s} = 0 \\
    \vec{M_2M} \vec{s_2} \vec{s} = 
    \begin{vmatrix}
      x - 2 & y & z + 4 \\
      -3 & 1 & 3 \\
      1 & 0 & 1
    \end{vmatrix} = 
    x + 6y - z - 6 \\
    \boxed{\alpha_2: x + 6y - z - 6 = 0}
  \end{gather*}

  5) Для начала, определим одну из координат точек. Прямая $l$ пересекает плоскость $oXY$, т.е. можем взять $z = 0$. Тогда в системе уравнений:
   \begin{gather*}
    \begin{cases}
      -3x + 4y + 3z - 4 = 0 \\
      x + 6y - z - 6 = 0
    \end{cases} \\
    \text{Полагаем, что } z = 0: \\
    \begin{cases}
      -3x + 4y - 4 = 0 \\
      x + 6y - 6 = 0
    \end{cases} \implies \begin{cases}
      x = 0 \\
      y = 1 
    \end{cases} \implies A(0, 1, 0)
  \end{gather*}
  где точка $A \in \alpha_1, \alpha_2, l$.
  
  Составляем каноническое уравнение прямой $l$, проходящей через точку $A$, и с направляющим вектором $\vec{s}$.
  \begin{gather*}
    \frac{x - 0}{1} = \frac{y - 1}{0} = \frac{z - 0}{1} \\
    \boxed{\frac{x}{1} = \frac{y - 1}{0} = \frac{z}{1}}
  \end{gather*}
\end{eg}

